\documentclass[12pt, a4paper]{article}
\usepackage[lmargin =0.75 in, 
			rmargin=0.75in, 
			tmargin=0.75in,
			bmargin=0.75in]{geometry}
\geometry{letterpaper}
\usepackage{amsmath}
\usepackage{amssymb}
\usepackage{blindtext}
\usepackage{titlesec}
\usepackage{enumitem}
\usepackage{fancyhdr}
\usepackage{amsthm}
\usepackage{graphicx}
\usepackage{cool}
\usepackage{thmtools}
\usepackage{hyperref}
\graphicspath{ }					%path to an image

%-------- sexy font ------------%
%\usepackage{libertine}
%\usepackage{libertinust1math}

%\usepackage{mlmodern}				% very nice and classic
\usepackage[utopia]{mathdesign}
%\usepackage[T1]{fontenc}


%\usepackage{mlmodern}
%\usepackage{eulervm}
%\usepackage{tgtermes} 				%times new roman
%-------- sexy font ------------%


% Problem Styles
%====================================================================%


\newtheorem{problem}{Problem}


\theoremstyle{definition}
\newtheorem{thm}{Theorem}
\newtheorem{lemma}{Lemma}
\newtheorem{proposition}{Proposition}
\newtheorem{fact}{Fact}
\newtheorem{definition}{Definition}
\newtheorem{example}{Example}

\newtheorem{manualprobleminner}{Problem}

\newenvironment{manualproblem}[1]{%
	\renewcommand\themanualprobleminner{#1}%
	\manualprobleminner
}{\endmanualprobleminner}

\newcommand{\penum}{ \begin{enumerate}[label=\bf(\alph*), leftmargin=0pt]}
	\newcommand{\epenum}{ \end{enumerate} }

% Math fonts shortcuts
%====================================================================%

\newcommand{\N}{\mathbb{N}}                           % Natural numbers
\newcommand{\Z}{\mathbb{Z}}                           % Integers
\newcommand{\R}{\mathbb{R}}                           % Real numbers
\newcommand{\C}{\mathbb{C}}                           % Complex numbers
\newcommand{\F}{\mathbb{F}}                           % Arbitrary field
\newcommand{\Q}{\mathbb{Q}}                           % Arbitrary field
\newcommand{\PP}{\mathcal{P}}                         % Partition
\newcommand{\M}{\mathcal{M}}                         % Mathcal M
\newcommand{\eL}{\mathcal{L}}                         % Mathcal L
\newcommand{\T}{\mathcal{T}}                         % Mathcal T
\newcommand{\U}{\mathcal{U}}                         % Mathcal U\\
\newcommand{\V}{\mathcal{V}}                         % Mathcal V

% symbol shortcuts
%====================================================================%

\newcommand{\lam}{\lambda}
\newcommand{\imp}{\implies}
\newcommand{\all}{\forall}
\newcommand{\exs}{\exists}
\newcommand{\delt}{\delta}
\newcommand{\eps}{\epsilon}
\newcommand{\ra}{\rightarrow}

\newcommand{\ol}{\overline}
\newcommand{\f}{\frac}
\newcommand{\lf}{\lfrac}
\newcommand{\df}{\dfrac}

% bracketting shortcuts
%====================================================================%
\newcommand{\abs}[1]{\left| #1 \right|}
\newcommand{\babs}[1]{\Big|#1\Big|}
\newcommand{\bound}{\Big|}
\newcommand{\BB}[1]{\left(#1\right)}
\newcommand{\dd}{\mathrm{d}}
\newcommand{\artanh}{\mathrm{artanh}}
\newcommand{\Med}{\mathrm{Med}}
\newcommand{\Cov}{\mathrm{Cov}}
\newcommand{\Corr}{\mathrm{Corr}}
\newcommand{\tr}{\mathrm{tr}}
\newcommand{\Range}[1]{\mathrm{range}(#1)}
\newcommand{\Null}[1]{\mathrm{null}(#1)}
\newcommand{\lan}{\langle}
\newcommand{\ran}{\rangle}
\newcommand{\norm}[1]{\left\lVert#1\right\rVert}
\newcommand{\inn}[1]{\lan#1\ran}
\newcommand{\op}[1]{\operatorname{#1}}
\newcommand{\bmat}[1]{\begin{bmatrix}#1\end{bmatrix}}
\newcommand{\pmat}[1]{\begin{pmatrix}#1\end{pmatrix}}
\newcommand{\vmat}[1]{\begin{vmatrix}#1\end{vmatrix}}

\newcommand{\amogus}{{\bigcap}\kern-0.8em\raisebox{0.3ex}{$\subset$}}
\newcommand{\Note}{\textbf{Note: }}
\newcommand{\Aside}{{\bf Aside: }}
%restriction
%\newcommand{\op}[1]{\operatorname{#1}}
%\newcommand{\done}{$$\mathcal{QED}$$}

%====================================================================%


\setlength{\parindent}{0pt}      	% No paragraph indentations
\pagestyle{fancy}
\fancyhf{}							% fancy header

\setcounter{secnumdepth}{0}			% sections are numbered but numbers do not appear
\setcounter{tocdepth}{2} 			% no subsubsections in toc

%template
%====================================================================%
%\begin{manualproblem}{1}
%Spivak.
%\end{manualproblem}

%\begin{proof}[Solution]
%\end{proof}

%----------- or -----------%

%\begin{problem} 		
%\end{problem}	

%\penum
%	\item
%\epenum
%====================================================================%


\newcommand{\Course}{STA302}
\newcommand{\hwNumber}{}

%preamble

\title{{\Course} notes}
\date{\today}

\chead{\Course}

\cfoot{\thepage}


%====================================================================%

\begin{document}
	
	\maketitle
	
	STA302F in Summer 2022 with Mohammad Khan. Feel free to email  \href{mailto://anton.sugolov@mail.utoronto.ca}{\bf anton.sugolov@mail.utoronto.ca} if there are any mistakes, or edit the tex \href{https://sugolov.github.io/blog/files/sta302.tex}{\bf here}.
	
	\tableofcontents

\newpage
	
	\section{May 9: Lecture 1}
	\subsection{Syllabus}
	
	This is a course on linear regression. The focus is using R to do data analysis, and build the mathematical foundation for regression. We will understand how prediction works later, which is the foundation for data science.\\
	
	{\bf Marking} 
	\begin{itemize}
		\item 2 HW - 15\% each, due June 1, June 15
		\item Test - 25\% on May 25
		\item Exam - 45\% during June 22-27
	\end{itemize}
	
	{\bf Books} J. Sheather, A Modern Approach to Regression w/ R and D. Montgomery, Linear Regression Analysis.
	
	\subsection{Review}
	
	\begin{definition}
		A {\bf sample space} $S$ is the set of possible events. 
		A {\bf random variable} is a function $X \colon S \to \R$ assigning a number to elements of the sample space.
	\end{definition}

	Constants can also be pseudo random variables. These are called {\bf degenerate random variables} that have a {\bf degenerate distribution} since they have infinite cdf.
	
	\begin{definition}
		For an event $A \subset S$, we define the {\bf indicator function} $I_A$ as
		$$
			I_A(s) = \begin{cases}
				1, & s\in A \\
				0, & s\notin A
			\end{cases}
		$$
	\end{definition}

	These are important since we later use them to create dummy variables in linear regression. When we write an inequality involving random variables, we mean that it holds for all elements of the sample space. I.e. $X \geq Y \imp X(s) = Y(s), \all s \in S$.\\
	
	\begin{example}
		Consider $S = \{1,2,3,4,5,6\}$. For $s \in S$, $X(s) = s$, let $Y(s) = X(s) + I_6(s)$. Then $Y = X$ for all $s \in S$ except $6$, where $Y = 7,\, X = 6$. 
	\end{example}
	
	
	\begin{definition}
		{\bf Discrete r.v.} are functions from a countable sample space, and {\bf continuous r.v.} are functions from an uncountable sample space. There are also {\bf mixture} random variables, which are continuous/discrete for different parts of the sample space. Random variables can be univariate and multivariate as well.
	\end{definition}	
	
	\begin{example}
		The multinomial distribution is an example of a discrete multivariate random variable.
	\end{example}
	
	\begin{definition}
		If $X$ is a random variable, the p.d.f. is the derivative of the c.m.f. As well, $\PP(a \leq X \leq b) = \int_a^b f(x)dx$ where $f(x)$ is pdf. Similar thing holds for discrete r.v.
	\end{definition}

	\begin{proposition}
		The expectation of two random variables is linear. For $Z = aX + bY$, $X,Y$ r.v., then $E(Z) = aE(X) + bE(Y)$.
	\end{proposition}

	\begin{definition}
		The {\bf variance} of $X$ is $V(X) = E(X - \mu_x)^2$. The {\bf sample variance} $s^2 = \f{\sum (x_i - \ol{x})^2}{n-1}$. Note we divide by $n-1$ so that it is an unbiased estimator (STA261).
	\end{definition}

	Some properties:
	\begin{itemize}
		\item $V(X) \geq 0$
		\item $V(aX + b) = a^2V(x)$
		\item $V(X) = E(X^2) - E(X)^2$
		\item $V(X) \leq E(X^2)$
		\item $\sigma_X = \sqrt{V(X)}$
	\end{itemize}
	\Note In linear regression, the variance of the predicted variable depends on the slope of regression line but not on the intercept (second property).\\
	
	Let $X_1, X_2, Y$ be r.v. and $A$ be an event. Let $Z = aX_1 + bX_2$. Then
	\begin{itemize}
		\item $E(Z \mid A) = aE(X_1 \mid A) + bE(X_2 \mid A)$
		\item $E(Z \mid Y = y) = aE(X_1 \mid Y = y) + bE(X_2 \mid
		Y = y)$
		\item $E(Z \mid Y ) = aE(X_1 \mid Y ) + bE(X_2 \mid
		Y)$
	\end{itemize}
	\begin{proposition}(Laws of Total Expecation and Variance)
		$E(E(Y \mid X)) = E(Y)$ and $V(X) = V(E(X \mid Y)) + E(V(X \mid Y))$.
	\end{proposition}

	We will see that linear regression is a conditional r.v., and the above will be very useful. For $X_1, \ldots, X_n$ i.i.d. random variables, $x_1 \ldots x_n$ realizations, then $\ol{x} = \f{\sum x_i}{n}$. The {\bf sample average} $\ol{X} = \f{\sum X_i}{n}$ is a random variable. In general, any function of $n$ i.i.d. random variables is a random variable, and called a {\bf sampling statistic} that follows a {\bf sampling distribution}.
	
	\begin{thm}(Central Limit Theorem)
		For $X_1, \ldots, X_n$ i.i.d. $f(x, \theta)$, $E(X), V(x) < \infty$, then $\f{\ol{X} - \mu}{\sigma / \sqrt{n}} \to N(0,1)$ converges in distribution for sufficiently large $n$.
	\end{thm}
	\begin{proof}
		Proof with moment generating functions.
	\end{proof}
	\begin{example}
		In the Cauchy distribution, this does not hold since it has infinite mean and variance.
	\end{example}
	\begin{definition}
		The {\bf covariance } $\Cov(X,Y) = E[(X - \mu_x)(Y - \mu_Y)] = E(XY) - E(X)E(Y)$. Covariance quantifies the relationship between two variables, i.e. how much one varies with the other. The {\bf correlation } $\Corr(X,Y) = \f{Cov(X,Y)}{\sqrt{V(X)V(Y)}}$.
	\end{definition}
	\begin{itemize}
		\item Covariance is an inner product, variance is norm.
		\item $V(X + Y) = V(X) + V(Y) + 2\Cov(X,Y)$.
		\item If $X \perp Y$, $V(X+Y) = V(X) + V(Y)$. 
		\item  In general, $V(\sum_i X_i) = \sum_i V(X_i) + 2 \sum_{i < j}\Cov(X_i,X_j)$.
	\end{itemize}
	These will be useful in regression, where we try to identify relationships between r.v.s.\\
	
	{\bf Definitions in statistics} \\ 
	
	In probability, we are given a mathematical model to work with. In statistics, we infer properties of a mathematical model. The steps of data analysis are: state the problem, identify what data is needed, decide on a model and collect data, clean data, estimate parameters of the model, and carry out appropriate tests, draw conclusions.
	
	\subsection{Introduction to Regression}
	
	\begin{definition}
		The {\bf corelation coefficient}
		$$
			\rho_{X,Y} = \f{\sum_i (x_i - \ol{x}) (y_i - \ol{y})}{\sqrt{\sum_i (x_i - \ol{x})^2} \sqrt{\sum_i (y_i - \ol{y})^2} } = \f{\Cov(X,Y)}{s_x s_y}
		$$
	\end{definition}

	The above value is somewhat like the $\cos(\theta)$ between the vectors $X,Y$; recall dot product. When we discuss corelation, we talk about linear relations only; the linear association between $X,Y$. We can see this by considering $X$ and $ Y = X^2$.	Corelation is symmetric, it does not indicate the direction of the symmetry (which causes which/causation). Corelation only says the influence on the change of one variable when the other changes; think about moving along non-orthogonal vectors and projecting.\\
	
	Galton investigated the effect of fathers heights on their sons height. Galton termed {\bf regression} as a `regression' of heights towards the mean; on average, heights of sons move towards the mean, so the average height across generations is the same.\\
	
	In a linear regression, we assume there is a linear relation $Y = \beta_0 + \beta_1 X + \eps$ between the random variables $X, Y$ where $\eps$ is an error random variable. The deviation not captured by linearity is incorporated to $\eps$. Given two values of $X$, it is not guaranteed that the value of $Y$ is the same. But for a unique $X$ we get {\bf unique average} $Y$. We want $E(Y \mid X = x) = \beta_0 + \beta_1 X$; the relationship between the mean of $Y$ and a specific value of $X$ is linear. Note $E(\eps) = 0$. We call $X$ the {\bf explanatory, predictor, independent} variable and $Y$ as the {\bf response, outcome, dependent} variable. Suppose we are given paired data $(x_1, y_1), \ldots, (x_n, y_n)$. We try to fit a linear regression to model the relationship between $X$ and $Y$:
	$$
		Y = \beta_0 + \beta_1 X + \eps	\text{ and want }	E(Y \mid X = x) = \beta_0 + \beta_1 X
	$$ 
	The values of $\beta_0, \beta_1$ are not yet known and need to be estimated. In the sample, the error $e_i$\ replaces $\eps_i$. The line best predicting $Y$ as $X$ changes should minimize the squares of the errors $e_i = y_i - \hat{y_i}$ where $\hat{y_i}  = b_0 + b_1 x_i$ where $b_0, b_1$ are the intercept and slope of the regression line. We minimize the squares $\sum_i e_i^2$. The $e_i$ are referred to as {\bf residuals}; minimize residual sums squared. Note
	$$
		RSS(b_0, b_1) = \sum_i e_i^2 = \sum_{i=1}^n (y_i  - \hat{y_i})^2 = \sum_{i=1}^n (y_i - b_0 - b_1 x_i)^2
	$$
	\Aside What value of $a$ minimizes (1) $\sum \abs{x_i - a}$, and which minimizes (2) $\sum (x_i - a)^2$? Answer: (1) $a = \Med(X)$, (2) $a = \ol{x}$. We do not minimize the sum of the residuals, since this must always be $0$. We minimize the RSS with respect to $b_0, b_1$.
		$$\pderiv[1]{RSS}{b_0} = -2\sum_i (y_i - b_0 - b_1 x_i),\, \pderiv[1]{RSS}{b_1} = -2\sum_i x_i(y_i - b_0 - b_1 x_i)$$
	so setting these to 0, we get the { \bf normal equations}
	$$
		\sum_{i} y_i = b_0n + b_1 \sum_{i} x_i,\ \ 	\sum_{i} x_iy_i =  b_0 \sum_{i} x_i + b_1 \sum_{i} x_i^2
	$$
	Solving these, we get
	$$
		\hat{\beta_0} = b_0 = \ol{y} - \hat{\beta_1}\ol{x},\qquad \	\hat{\beta_1} = b_1 = \f{\sum_i x_i y_i - n\ol{x}\ol{y}}{\sum x_i^2 - n\ol{x}^2} = \f{\sum(x_i - \ol{x})(y_i - \ol{y})}{\sum(x_i - \ol{x})^2} = \f{S_{X,Y}}{S_X}
	$$
	The intercept is the average value of the response when $X=0$.
	
	\subsection{Class Afterthoughts/Questions}
	When the errors have $E(\eps = 0)$, then $V(\eps) = E(\eps^2) - E(\eps)^2 = E(\eps^2)$. By minimizing this in the sample, we minimize the variance of the errors (?)
	
	\section{May 11: Lecture 2}
	{\bf Clarifying last class:} $\hat{y_i}$ is the conditional mean of $y_i$. When this is true, then $\sum_i e_i = 0$. That is, we estimate $\hat{y_i}$ so that $\sum_i e_i = 0$. 
	
	\subsection{Regression continued}
	We continue discussing linear regression; fitting a linear relation assuming it exists. The aim is to infer the true values of $\beta_0, \beta_1$ by inspecting their sampling distributions. We also make some assumptions regarding the error terms; the properties of their distributions ($\eps$ is r.v.).
	
	\subsubsection{Assumption: Linearity}
	The conditional mean of $Y \mid X = x$ is linear with respect to $X$. However, the relationship $E(Y \mid X)$ and $X$ does not have to be linear, but the linearity assumption is linearity in the parameters.Our relationship must be realistic given the context; introducing linearity may produce unrealistic relationships.\\
	
	{\bf R simulation: } When generating random dataset, we set a seed so our results are reproducible. Always start with a seed in assignments. Note the $Y$ variable is the transformation $\beta_0 + \beta_1 \log X + \eps$. Introducing linear relationship between $X$ and $Y$ is inaccurate. It is linear in the parameters $\beta_0, \beta_1$ however.\\
	
	{\bf Qs:} Chaos in random number generation? Look up random number generation algorithms. How do we quantify linearity in a data set? Mostly with plots but is there better way?
	
	\subsubsection{Assumption: Independence}
	The errors $\eps_i$ are independent. That is, the deviations from the mean are not related; they are i.i.d. r.v. This reduces predictive capibilities in some areas, but we can relax this assumption later (generalized least squares).
	
	\subsubsection{Assumption: Homoscedasticity (equal variance)}
	The error variance does not change depending on $X$. That is $V(\eps \mid X = x) = \sigma^2$ and is independent of $x$. In the R codes, we see that variance of errors increases with $X$, which decreases predictive power as $X$ increases. Moreover, this implies some of the variation in the errors is explained by $X$, which violates our assumption. Variance {\bf cannot} depend of $X$. $\eps \perp X$. This is relaxed in GLS.\\
	
	 In multiple linear regression, we talk about the Gauss-Markov assumption, but we need to make some assumptions about how $\eps_i$ is distributed in order to make inferences.
	
	\subsubsection{Assumption: Normality}
	$\eps \sim N(0, \sigma^2)$. The previous assumptions are required to obtain the least squares estimates, but normality is not required. Under this assumption, we can make confidence intervals and tests, and have nice properties following from normal distribution.\\
	
	There are more assumpitons in general, but these are most important.
	
	\subsubsection{More about variance of $\eps$}
	We have estimated $\beta_0$, $\beta_1$ using least squares. However, we have another parameter to estimate; $V(\eps) = \sigma^2$. From afterthoughts, $V(\eps) = E(\eps^2) = \sigma^2$. We take the average of $e_i^2$ using this, since we want summary measure. The mean residual squared (MRS) can be calculated as $s^2 = \df{\sum_i e_i^2}{n-2}$. We show this estimator of $E(\eps^2)$ is unbiased as homework; prove this!.
	
	\subsection{Inferences about the regression model}
	
	\subsubsection{Conditional expectation and variance of $\hat{\beta_1}$}
	Recall $\beta_1 = \df{\sum(x_i - \ol{x})(y_i - \ol{y})}{\sum(x_i - \ol{x})^2}$
	\begin{proposition}
		$\sum(x_i - \ol{x})(y_i - \ol{y}) = \sum_i (x_i - \ol{x}) y_i$
	\end{proposition}
	\begin{proof}
		\begin{align*}
			\sum(x_i - \ol{x})(y_i - \ol{y}) &=\sum ( x_i y_i - \ol{x} y_i - \ol{y} x_i + \ol{x}\ol{y}) \\
			&= \sum ( x_i y_i - \ol{x} y_i) - n \ol{y} \ol{x} + n\ol{x}\ol{y} \\
			&= \sum ( x_i - \ol{x})y_i
		\end{align*}
	\end{proof}
	A symmetric sum can be established for $\sum_i (y_i - \ol{y}) x_i$. However, the above is needed to simplify conditional expectation calculations. We may also show $\sum ( x_i - \ol{x})x_i = \sum (x_i - \ol{x})^2$. The idea of both of these proof is making the substitution $n\ol{x} = \sum x_i$.
	\begin{proposition}
		$\sum ( x_i - \ol{x})x_i = \sum (x_i - \ol{x})^2$
	\end{proposition}
	\begin{proof}
		\begin{align*}
			\sum (x_i - \ol{x})x_i &=  \sum (x_i^2 - \ol{x}x_i)\\
			&= \sum (x_i^2 - 2\ol{x}x_i) + n\ol{x}^2\\
			&= \sum (x_i - \ol{x})^2
		\end{align*}
	\end{proof}
	Other way of writing: $\sum (x_i \ol{x})^2 = \sum x_i^2 - n\ol{x}^2$. Now, we calculate {\bf conditional expectation of $\hat{\beta_1}$}
		$$
			E(\hat{\beta_1} \mid X = x_i) = E\BB{\f{
					\sum(x_i - \ol{x})y_i}{
					\sum(x_i - \ol{x}^2)
				}
			\mid X = x_i
			} 
		= \f{
			\sum (x_i - \ol{x})E(Y_i \mid X = x_i)
		}{
			\sum (x_i - \ol{x})^2
		}
		$$
		Substituting $E(Y_i \mid X_i = x) = \beta_0 + \beta_1 x$, then
		$$
			E(\hat{\beta_1} \mid X = x_i) = 
		\f{
				\sum_i (x_i - \ol{x}) \beta_0 
		}{
			\sum (x_i - \ol{x})^2
		} 
		+ \f{
			\sum_i (x_i - \ol{x}) \beta_1 x_i
		}{
			\sum (x_i - \ol{x})^2
		} = \f{
			 \beta_1 \sum_i (x_i - \ol{x})^2
		}{
			\sum (x_i - \ol{x})^2 
		} = \beta_1
		$$
		Since $\sum(x_i - \ol{x}) = \sum x_i - n \ol{x} = 0$ and by above prop., $\sum_i (x_i - \ol{x}) x_i = \sum (x_i - \ol{x})^2 $. Therefore $\hat{\beta_1}$ does not depend on $X$, and has expected value of $\beta_1$; it is an unbiased estimator of $\beta_1$. That is, $	E(\hat{\beta_1} \mid X = x_i) = 	E(\hat{\beta_1} ) = \beta_1$. Next, we may calculate $V(\hat{\beta_1})$. First, $V(Y_i \mid X = x_i) = \sigma^2$, that is, the variance of the error.
		\begin{align*}
			V(\hat{\beta_1} \mid X = x_i) = \BB{\f{
					\sum(x_i - \ol{x})y_i}{
					\sum(x_i - \ol{x}^2)
				}
				\mid X = x_i
			} 
			= \f{
				\sum_i (x_i - \ol{x})^2V(Y_i \mid X = x_i)
			}{
				(\sum_i (x_i - \ol{x})^2)^2
			} 
			= \f{
			\sigma^2
			}{
			\	\sum (x_i - \ol{x})^2
			}
			=	\f{\sigma^2}{S_{X,X}}
		\end{align*}
	
		\subsubsection{Inferences for variance of $\hat{\beta_1}$}
	
		Since $\eps_i \sim N(0, \sigma^2)$, then $Y_i \mid X \sim N(\beta_0 + \beta_1 X, \sigma^2)$. Letting $c_i = \f{\sum(x_i - \ol{x})}{\sum(x_i - \ol{x})^2}$ then $\hat\beta_1 = \sum c_iy_i$. Observe that this is a {\bf linear combination} of normally distributed random variables, so $\hat\beta_1$ is normally distributed! Thus
		$$
			\hat\beta_1 \mid X = x_i \sim N\BB{\beta_1, \f{\sigma^2}{S_{X,X}}}
		$$
		We can construct a $1-\alpha$ confidence interval for $\beta_1$ which has extremes $\hat\beta_1 \pm Z_{1-\alpha/2} \df{\sigma}{\sqrt{S_{X,X}}}$. When $\sigma^2$ is unknown, we construct a $t$-confidence using $S^2 = \df{\sum e_i^2}{n-2}$.
		We therefore make a confidence interval with critical values
		$$
			\hat\beta_1 \pm t_{1-\alpha/2, n-2} \df{s^2}{\sqrt{S_{X,X}}}
		$$
		Note our assumption of normality of errors.\\
		
		{\bf Clarification} $S_{X,X} = \sum (x_i - \ol x)^2$ and $S_{X,Y} = \sum (x_i - \ol x)(y_i - \ol y)$.\\
		
		Recall, the {\bf p-value} can be calculated as $p = \PP(Z \geq \abs{z})$ or $p = \PP(T \geq \abs{t})$ where $z,t$ are the calculated test statistics. The p-value is the probability of obtaining a sample that provides strong evidence against the hypothesized value of $H_0: \beta_1$, set by threshold $\alpha$. $\alpha$ is the probability of making a type one error with repeated sampling.
		
		\begin{example}
			$\sum x_i = 4035,\, \sum y_i = 4041,\, \sum e_i^2 = 4753.125,\, \sum x_i^2 = 1005535,\, \sum x_iy_i = 864910,\, t_{0.975, 18} =2.10$.
			
			We need to calculate $\hat\beta_1, s, S_{X,X}$ from this information; recall $\hat\beta_1 \pm t_{1-\alpha/2, n-2} \df{s}{\sqrt{S_{X,X}}}$. The interval becomes $(0.18121, 0.33728)$. {\bf Verify as homework.}
		\end{example}
		Do exercises from Montgomery (unassigned, do by chapter) and Sheather. Problems are similar to this, and this will appear on the midterm.
		
		\subsubsection{Properties of $\beta_0$}
		
		The conditional expectation of $\beta_0 \mid X$. Since $\hat\beta_0 = \ol y - \hat \beta_1 \ol x$. Using this,
		$$
			E(\hat\beta_0 \mid X = x_i) = \f{\sum E(y_i \mid X = x_i)}{n} - \beta_1 \ol x = \BB{\f{n\beta_0 + n\beta_1 \ol x}{n}} - \beta_1 \ol{x} = \beta_0
		$$
		Therefore $\hat \beta_0$ is an unbiased estimator of $\beta_0$. Now for the variance, (minor abuse of notation)
		$$
			V(\hat\beta_0 \mid X = x_i) = V(ol y - \hat \beta_1 \ol x \mid X = x_i) = V(\ol y \mid x_i) + \ol x^2 V(\hat\beta_1 \mid x_i) - 2 \ol x \Cov(\ol{y}, \hat\beta_1 \mid x_i)
		$$
		Calculating each term separately,
		$$
			V(\ol y \mid X = x_i) = V\BB{\f{\sum y_i}{n} \mid X = x_i} =  \f{\sum \sigma^2}{n^2} = \f{\sigma^2}{n}
		$$
		To calculate covariance term, we use substitutions involving $\hat\beta_1 = \sum c_iy_i$ with $c_i$ defined before
		\begin{align*}
			\Cov(\ol{y}, \hat\beta_1 \mid X =  x_i) &= \Cov\BB{\f{\sum_i y_i}{n}, \sum c_iy_i \mid X = x_i} 
			= \f1n \sum_i \Cov(y_i, c_iy_i \mid X = x_i)
			\intertext{Recall $\Cov(X, aY) = a \Cov(X,Y)$. Also, given a particular $x_i$, $c_i$ is a constant.}
			&= \f1n \sum_i c_i \Cov(y_i, y_i \mid X = x_i) =  \f1n \sum_i c_i V(y_i \mid X = x_i) = \f1n \sum_i c_i \sigma^2 = 0
		\end{align*}
		From last section, $V(\hat\beta_1 \mid x_i) = \ol x^2 \f{\sigma^2}{S_{X,X}}$. Therefore
		$$
		V(\hat\beta_0 \mid X = x_i) = \sigma^2 \BB{\f1n + \f{\ol x^2}{S_{X,X}}}, \text{ and } \hat\beta_0 \mid X = x_i \sim N\BB{\beta_0, \sigma^2 \BB{\f1n + \f{\ol x^2}{S_{X,X}}}}
		$$
		Therefore the $(1-\alpha)$ confidence for $\beta_0$ is
		$$
			\hat \beta_0 \pm Z_{1 - \alpha/2} \sigma \sqrt{\f1n + \f{\ol x^2}{S_{X,X}}} 
		$$
		(fill in when $\sigma^2$ is unknown )
		
		\subsubsection{Confidence interval for the regression line}
		
		Denote $x^*, y^*$ as an observation not currently in the sample. We use the model built with the current observations to see how far $y^*$ observation can vary. It can easily be shown that 
		$$E(\hat y^* \mid X = x^*) = \beta_0 + \beta_1 x^*$$
		Where $X = x^*$ new observation, $y^*$ unknown. As well, $\hat y^*$ is the predicted value of $y^*$ paired with $x^*$. Often, we are interested in calculating the variance of $E(Y \mid X = x^*) = \hat y^* \mid X = x^*$ and confidence interval for $E(Y \mid X = x^*)$. That is, calculating the variance and confidence of the regression line at each point. Note $E(\hat y^* \mid X = x^*) = \beta_0 + \beta_1 x^* = E(Y\mid X = x^*)$ implies the sample regression is an unbiased estimator of the true Linear relationship between $X,Y$. The variance can be calculated as 
		\begin{align*}
			V(\hat y^* \mid X = x^*) &= V(\hat \beta_0 + \hat \beta_1 x^*) = V(- \ol y + \hat\beta_1(x^* - \ol x))  \\
			&=  V(\ol y) + (x^* - \ol x)^2V(\hat \beta_1) = \f{\sigma^2}{n} + \f{\sigma^2(x^* - \ol x)^2 }{S_{X,X}} \\
			&=  \sigma^2 \BB{ \f1n + \f{(x^* - \ol x)^2}{S_{X,X}} }
		\end{align*}
		This is interpreted as the variance of the true location of the regression line at $X = x^*$. Note variance increases quadratically as $x^*$ moves further from $\ol x$.
		
		\subsubsection{Prediction error and interval}
		Assuming we fit a regression line between $X,Y$ with some sample. If a new data point $X = x^*$ is given, our predicted $\hat y^*$ lies exactly on the line in the model we have fitted, but $y^*$ associated with $x^*$ may deviate from the line. How much does this $y$ vary? $y^* - \hat y^*$ is called the {\bf prediction error} for $X = x^*$. We calculate its expectation and variance.\\
		
		For expectation, the $*$ is redundant, so we write
		$E(y - \hat y \mid X = x^*)$. We can easily show this is $0$ since $y - \hat y = 0$.
		Therefore 
		$$
			V(y^* - \hat y^* \mid X = x^*) = V(y - \hat y \mid X = x^*) = \sigma^2 \BB{
				1 + \f1n + \f{(x^* - \ol x)^2}{S_{X,X}}
			}
		$$
		We just add the variance of $y$ and variance of $\hat y$ by expansion of variance and since $\Cov(\hat y, y) = 0$. The observation $y$ is independent of the previous sample by assumption. The prediction interval is built in the same way as before using $t$ distribution. The prediction interval is how much we expect the true value to deviate from the regression line.
		
		{\bf R simulation:} 
		
		The confidence interval is for the regression line. The prediction interval is for a new predicted value given $x^*$; how far $y^*$ can deviate from the predicted $\hat y^*$.
		
		\begin{example}
			Calculate summary measures for the production data (in slides hw)
		\end{example}
	
		\section{May 16: Lecture 3}
		
		{\bf Clarification } In the derivations from last class, we used
		$$
			\Cov \BB{ \f{\sum Y_i}{n}, \sum c_i y_i \mid X = x_i} = \f1n \sum \Cov(y_i, c_i y_i \mid X = x_i)
		$$
		since $\Cov(Y_i, Y_j) = 0$ by independence of $Y_i, Y_j$.\\
		
		Understand theory and problem solving procedure for midterms. Data analysis will mostly be with R.		
		
		\subsubsection{Assignment Task 1}
		The purpose of the assignment is using R for inference of parameters given simulated data. Use your student id as a seed. After data is generated, run the LM model. Repeating this procedure, get sampling distribution for $\hat\beta_i, \sigma^2$, and compare these to true variances. 
		
		\subsection{Analysis of variance (ANOVA)}
		
		So far we have discussed inference about specific parameters, and hypothesis testing for their true values. For example, if we fail to reject $H_0 : \beta_1 = 0$, then there is no linear relationship between $X,Y$. In this case, $Y = \beta_0 + \eps$, $V(Y) = V(\eps) = \sigma^2$, so $\eps$ explains all the variance of $Y$. Usually, $V(Y) = \beta_1^2 V(X) + \sigma^2$, since $X \perp \eps$. Therefore when the above holds, part of the variance is given by $V(X)$. If most of the variation in $Y$ is explained by $X$, then predictions are very accurate. We discuss this in ANOVA. \\
		
		In the slides, points that are less scattered about the regression line have more of their variance explained by $X$.\\
		
		As the residual variance $\sigma^2$ increases, the variation of $Y$ is less explained by $X$. This increases prediction error. We want to answer how well the regression line might explain the variation we observe in the responses. ANOVA is another way of testing the significance of the regression line. The total varation of $Y$ is explained by the {\bf total sum of squares}, the numerator of $s_Y$
		$$
			SST = \sum (y_i - \ol y)^2
		$$
		This can be decomposed by 
		\begin{align*}
			\sum (y_i - \ol y)^2 &= \sum (y_i - \hat y_i + \hat y_i - \ol y)^2 = \sum (y_i - \hat y_i)^2 + \sum (\hat y_i - \ol y)^2 + 2 \sum (y_i - \hat y_i)(\hat y_i - \ol y)
		\end{align*}
		Where the third term becomes 
		\begin{align*}
			\sum (y_i - \hat y_i)(\hat y_i - \ol y) &= \sum (\hat y_i (y_i - \hat y_i) - \ol y(y_i - \hat y_i)) = \sum \hat y_i e_i - \ol y \sum e_i = 0
		\end{align*}
		Since $\sum e_i = 0$ and $\sum x_i e_i = 0$ by the second normal equation, which gives $\sum \hat y_i e_i = 0$. Hint: $\sum (\beta_0 + \beta_1 x_i)e_i = \beta_0 \sum e_i + \beta_1 \sum x_i e_i$.
		Therefore the total variation of $Y$ can be divided into 
		$$
			\sum (y_i - \ol y)^2  = \sum (y_i - \hat y_i)^2 + \sum (\hat y_i - \ol y)^2
		$$
		The term on the left is the {\bf residual sum square}, $(n-2)s^2$. The second term explains the variance in $\hat y_i$, or the variation in fitted values from the regression. We may easily show $\sum \f{\hat y_i}{n} = \ol y$. The second term on the right is the {\bf regression sum squared}. The total variation in $Y$ has been decomposed to come from the regression line, and from random errors.\\
		
		{\bf Degrees of Freedom.} This is the number of summed square normals. The proof for $\f{(n-1)s^2}{\sigma^2} \sim \chi_{n-1}^2$ shows where one of the `standard normal squares' are lost. ($s^2$ is sample variance). For each parameter we fix, we lose a degree of freedom. When $\ol y$ is fixed, we are free to have $n-1$ values, and are forced to choose one to get the fixed $\ol y$. That is, $y_n$, the $n$-th observation is fixed for a fixed $\ol y$. This is why sample variance, $\sum (y_i - \ol y)^2 / n-1$, uses $n-1$ degrees of freedom.\\
		
		In the above SST, the {\bf RSS} $\sum (y_i - \hat y_i)^2 $ has $n-2$ degrees of freedom since $\hat y_i = \hat \beta_0 + \hat \beta_1 x_i$ uses two estimated parameters. Since $\sum (y_i - \ol y)^2$ has $n-1$ degrees of freedom, then the $SS_{reg}$ $\sum (\hat y_i - \ol y)^2$ must have 1 degree of freedom. This follows since the sum depends only on $\beta_1$ given fixed $x_i$:
		\begin{align*}
			\sum (\hat y_i - \ol y)^2 
			= \sum (\hat \beta_0 + \hat \beta_1 x_i - \ol y^2) 
			= \sum (\ol y - \hat \beta_1 \ol x + \hat \beta_1 x_i - \ol y^2) 
			= \sum \hat \beta_1^2 (x_i - \ol x)^2
		\end{align*}
		We need degrees of freedom in order to test hypothesis. We will later show
		$$
			\f{SS_{reg}}{\sigma^2} \sim \chi_1^2,\, \f{RSS}{\sigma^2} \sim \chi_{n-2}^2
		$$
 		Under $H_0: \beta_0 = 0$ then $F_0 \sim F_{1, n-2}$. We want $SS_{reg}$ as close to the SST as possible. The F-test here detects how close $SS_{reg}$ is to TSS. The closer it is the bigger the value of $F_0$. We can show $t^2_{n-2} = F_{1, n-2}$. We can also  show
 		$$
 			E(SS_{reg}) = \sigma^2 + S_{X,X}\beta_1^2
 		$$
		So when $\beta_1 = 0$, the regression sum squared have variance equal to $\sigma^2$. Below is an ANOVA table:
		
		\begin{table}[ht]
			\centering
			\begin{tabular}{lrrrrr}
				\hline
				Sources of Variation & Df & Sum Sq & Mean Sq & F value & Pr($>$F) \\ 
				\hline
				Regression & 1 & $SS_{reg}$ & $MS_{reg} = \f{SS_{reg}}{1}$  & $F_0 = \f{MS_{reg}}{MRSS}$ & etc \\ 
				Residuals & n-2 & $RSS$ & $MRSS_{reg} = \f{RSS}{n-2}$ &  &  \\ 
				\hline
				Total & n-1 & SST &&&
			\end{tabular}
		\end{table}
		In general, the F-test measures whether the means of two groups measure significantly. The F statistic is the ratio of explained variance (regression model attributes to $V(X)$) to unexplained variance (variance of $e_i$). Under the null, our data reflects the intercept only model $Y = \beta_0 + \eps$, and we test the departure from this.
		
		\subsubsection{The Coefficient of Determination}
		
		Another measure to assess whether the regression line explains enough of the variability in the response is the {\bf coefficient of determination, $R^2$}. This gives the proportion of the total sample variability in the response that has been explained by the regression model.
		$$
			R^2 = \f{SS_{reg}}{SST} \text{ or } 1- R^2 = \f{RSS}{SST}
		$$
		Note $0 \leq R^2 \leq 1$. If $R^2$ is close to 1, it is an important predictor of $Y$. If it is close to 0, then it offers little predictive power for $Y$. In simple linear regression, $\rho^2 = R^2$ where $\rho$ is Pearson corelation coefficient.
	
		\subsubsection{Categorical predictors}
		
		So far we have required $X$ to be continuous. However, $X$ could be categorical. ($X$ smoking status vs. $Y$ blood pressure). Here the predictor is binary and the output is continuous. How would we test if the mean blood pressure varies between these groups?\\
		
		We did this in STA261 with a two-sample t-test, and by homoescadicity we do one with equal variance. We may also use regression, by using {\bf dummy variables} which are indicator variables. Setting $0$ for non-smokers, $1$ for smokers,
		$$
			E(Y \mid X = 0) = \beta_0,\, E(Y \mid X = 1) = \beta_0 + \beta_1
		$$
		Using ANOVA this is essentially a t-test. $F_{1,n-2} \sim t^2_{n-1}$ so by squaring the $t$ statistic we get $F$ statistic; a significant $F$ statistic indicate the change in means given by $\beta_1$ is significant. Therefore using hypothesis test with ANOVA for $\beta_1 = 0$, we get a test for differing means.\\
		
		The `slope' becomes the change in average. We can say $\beta_1$ reflects the average difference between two groups. The slope provides the magnitude of the difference, while the hypothesis test tells us whether the difference is statistically significant. \\
		
		With categorical variables, $R^2$ may be low but the test will give significance.
		
		\subsection{Multiple Linear Regression}
		
		So far we have only had one predictor $X$, but we generalize to $X_1, \ldots, X_n$. That is
		$$
			Y = \beta_0 + \beta_1 X_1 + \ldots + \beta_p X_p + \eps
		$$
		This implies $Y$ is related to $X_1, \ldots X_p$ linearly. However, the predictor produces a $p$-dimensional subspace instead of a line. See image in `Elements of Statistical Learning 2e'; with $Y$ regressed on $ X_1, X_2$ we get a regression plane.\\
		
		The conditional mean of $Y$ is given by $E(Y \mid X_1, \ldots, X_p) = \beta_0 + \beta_1 X_1 + \ldots + \beta_p X_p $. For the sample dataset, 
		$$
			y_i = \beta_0 + \beta_1x_{i,1} + \ldots + \beta_p x_{p,1} + e_i
		$$
		So we minimize $RSS(\beta_0, \ldots, \beta_p) = \sum (y_i - \sum^p \beta_j x_{ij})^2$. Differentiating with respect to each $\beta_j$,
		$$
			\pderiv[1]{RSS}{\beta_0} = \sum -2 (y_i - \sum^p \beta_j x_{ij}) 
			\qquad 
			\pderiv[1]{RSS}{\beta_j} \sum -2 (y_i - \sum^p \beta_j x_{ij})x_{ij}
		$$
		Setting these to $0$, we get $p+1$ normal equations in $p+1$ unknowns, giving us a unique solution and therefore minimum, since it is the minimum for each $\beta_j$.
		
		\subsubsection{Matrix Notation}
		In order to simplify notation we use matrices. For this we write 
			$$\mathbf{Y} = \mathbf{X} \beta + \eps$$
		$\mathbf{Y}$ is an $n \times 1$ vector, $\mathbf{X}$ is an $n \times (p+1)$ matrix, with the first column being a vector of $1$s. $\beta$ is $(p+1)\times 1$ vector, $\eps$ is $n \times 1$ vector.\\
		
		We denote the transpose of matrix $\mathbf{A}$ as $\mathbf{A'}$. If $\mathbf{A}$ is a square matrix with $\mathbf{A} = \mathbf{A'}$ then it is symmetric (corrseponds to self adjoint operator). If $\mathbf{A}$ is invertible, we denote its inverse with $\mathbf{A^{-1}}$. A matrix is {\bf orthogonal} if $\mathbf{A^{-1}} = \mathbf{A'}$; column vectors are orthogonal. An {\bf idempotent} matrix satisfies $\mathbf{A^2} = \mathbf{A}$. Some important properties are that $$(\mathbf{A}+\mathbf B)' = \mathbf A' + \mathbf B'\, \text{ and } (\mathbf A \mathbf B)' = \mathbf B' \mathbf A'$$
		
		\begin{example}
			The projection matrix $P : \R^n \to \R^n$ of rank $p \leq n$ onto a subspace is a square matrix that is symmetric and idempotent. 
		\end{example}
		
		\section{May 18: Lecture 4}
		
		\subsection{More properties}
		\begin{definition}
			If $Y = (Y_1, \ldots, Y_n)$ is a random vector, then 
			$
				E(Y) = (E(Y_1), \ldots, E(Y_n))
			$. The {\bf covariance matrix} of $Y$ is denoted
			$$
				V(Y) = 
				\pmat{ 
				V(Y_1) & \Cov(Y_1, Y_2) & \ldots & \Cov(Y_1, Y_n) \\
				\Cov(Y_2, Y_1) & V(Y_2) & \ldots & \Cov(Y_2, Y_n) \\
				\vdots && \ddots & \vdots \\
				\Cov(Y_n, Y_1) & \Cov(Y_n, Y_2) & \ldots & V(Y_n)
				 }
			$$ That is each entry $a_{i,j} = \Cov(Y_i, Y_j)$. It is created by $\Cov \{(Y - E(Y))(Y-E(Y))'\}$, the outer product. 
		\end{definition}
		
		\begin{proposition}
			If $A$ is a constant matrix, $X$ a random vector, then $E(AX) = A E(X)$
		\end{proposition}
		\begin{proposition}
			If $b$ is a constant vector, $Y$ a random vector, then $V(b'Y) = b'V(Y)b$.
		\end{proposition}
		
		\subsection{Multiple Linear Regression Continued}
		Above, we wrote $\mathbf{Y} = \mathbf{X} \beta + \eps$, that is  $y_i = \beta_0 + \beta_1 x_{i, 1} + \ldots + \beta_p x_{i, p} + \eps_i$ in matrix form. Explicitly,
		$$
		\pmat{
			y_1 \\ y_2 \\ \vdots \\ y_n	
			} =
		\pmat{1 & x_{1,1} & x_{1,2} & \ldots & x_{1,p} \\
			1 & x_{2,1} & x_{2,2} & \ldots & x_{2,p} \\
			\vdots && \ddots && \vdots \\
			1 & x_{n,1} & x_{n,2} & \ldots & x_{n,p} 
		} \pmat{ \beta_0 \\ \beta_1 \\ \vdots \\ \beta_p} + \pmat{\eps_1 \\ \eps_2 \\ \vdots \\ \eps_n}
		$$
		$\mathbf Y, \eps \in \R^n, \beta \in \R^{p+1}$, and $\mathbf X$ is $n \times (p+1)$ dimensional.\\
	
		As before, we would like to minimize $\sum_i^n e_i^2$ given values in $X$. This evaluates to the scalar
		$$RSS(\beta) = \sum_i^n e_i^2 = e'e = (Y - X\beta)'(Y - X \beta) = Y'Y - 2Y'X\beta + \beta'X'X\beta$$
		Where $Y'X\beta = \beta'X'Y$ since the transpose of a scalar is the same scalar. Note $RSS : \R^{p+1} \to \R$ Differentiating with respect to $\beta$,
		$$
			\pderiv[1]{RSS}{\beta} = \pderiv[1]{}{\beta} (Y'Y - 2\beta'X'Y + \beta'X'X\beta) = -2X'Y + 2X'X \beta
		$$
		Setting this to $0$, we see $\hat \beta = (X'X)^{-1}X'Y$. In the case of simple LR, 
		$$
			X = \pmat { 1 & x_1 \\ \vdots & \vdots \\ 1 & x_n},\, Y = \pmat{y_1 \\ \vdots \\ y_n} \imp 
			X'X = \pmat{ n & \sum x_i \\ \sum x_i & \sum x_i^2} = n \pmat{ 1 & \ol x \\ \ol x & \f1n \sum x_i^2 }
		$$
		We can compute $\det X'X = n^2 \cdot \BB{\df1n \sum x_i^2 - \ol x^2} = n \cdot \sum (x_i - \ol x)^2 = n \cdot S_{X,X}$. Therefore 
		$$
			(X'X)^{-1} = \pmat{
					\df{\sum x_i^2}{n \cdot S_{X,X}} & - \df{\ol x}{S_{X,X}} \\
					- \df{\ol x}{S_{X,X}} & \df{1}{S_{X,X}}
						}
		$$
		Multiplying by $\sigma^2$, we see this is the {\bf covariance matrix for $\hat\beta_0, \hat\beta_1$; $\Cov(\hat\beta_0, \hat\beta_1) = \df{-\sigma^2 \ol x}{S_{X,X}}$}. {\bf Important for midterm!}
		\begin{definition}
			The {\bf projection} of $Y$ on $X$ is given by $\hat Y = X \hat \beta = X(X'X)^{-1}X'Y = H Y$. We call $H$ the {\bf hat} or {\bf projection} matrix. Note it is $n \times n$, idempotent, and symmetric!
		\end{definition}
		We let $e = Y - \hat Y = Y - X(X'X)^{-1}X'Y = (I - H) Y$
		\begin{proposition}
			$H$ and $I-H$ are both idempotent.
		\end{proposition}
		Note that $HX = X$; this is easily checked by tracing definition and cancelling inverses. We can partition the first $k$ and last $p+1 - k$ columns of $X$ into matrix $[X_1, X_2]$. Then $HX = [HX_1, HX_2] = X = [X_1, X_2]$. As well, $\tr(H) = p+1$ and $\dim \text{range} H = p+1$. 
		
		\subsubsection{Assumptions in Multiple LR}
		
		$E(Y \mid X) = X \cdot \beta$. Linearity, independence, homoescadicity, normality hold as assumptions for our model (same as before). We assume $\eps \sim N(0, \sigma^2 I)$. Then $Y \mid X \sim N(X \beta , \sigma^2 I)$. Now we discuss the distribution of $\hat\beta$.
		$$
			E(\hat \beta \mid X) = E((X'X)^{-1}X'Y \mid X) = (X'X)^{-1}X'X\beta = \beta
		$$
		so the estimator is consistent. For the variance, we carry out adjoints as in previous property
		$$
			V(\hat \beta \mid X) 
			= V((X'X)^{-1}X'Y \mid X) 
			= (X'X)^{-1}X' \sigma^2 I X (X'X)^{-1}
			= (X'X)^{-1} \sigma^2
		$$
		This is just the covariance matrix of $\hat \beta$! Look back to our example above. That is 
			$$C = (X'X)^{-1} \imp c_{ij} = \sigma^2 \Cov(\beta_i, \beta_j)$$
		Least squares estimates are the {\bf best linear unbiased estimators} according to the Gauss-Markov Theorem (which is stated later). The following assumptions are required for the theorem: (1) the errors $\eps_i$ are independent, (2) $E(\eps) = 0$, (3) $V(\eps) = \sigma^2$. Note normality is {\bf not} assumed. \\
		
		As in simple LR, the $\hat \beta_j$ are normally distributed; $\hat \beta_j \sim N(\beta_j, \sigma^2 c_{j,j})$. We can test hypotheses for $\beta_j$ in the usual way. Given $H_O : \beta_j^0$, then we can calculate $Z = \f{\hat\beta_j - \beta_j^0}{\sqrt{c_{j,j}} \sigma }$ and use a z-test. 		
		
		
		
	\section{May 30: Lecture 5}
	
		\subsubsection{Term Test}
		
		Higher than expected. Expect lots of multiple linear regression questions in the final, like Question 5 on TT. Practice from Chapter 3 in Montgomery.
		
		\subsection{ANOVA for Multiple Linear Regression}
		
		\subsubsection{Expectation of RSS, sample variance}
		
		The RSS for MLR is $\sum(y_i - \hat y_i ) = e'e$. Recall $e = (I-H)y$ since $Y - \hat Y = Y - H Y = (I-H)Y$, where $H = X(X'X)^{-1}X'$.	Therefore 
			$$
				RSS = y' [ I - X(X'X)^{-1}X'] y =  y' [ I - H] y
			$$
		In MLR, we have $p+1$ parameters to estimate so reasoning with degrees of freedom, the {\bf sample variance} $s^2 = \df{RSS}{n-p-1} = \df{\sum e_i^2}{n-p-1}$. We show this by first calculating expectation of RSS by proving a theorem, and substituting $A = I-H$. Please see last lecture for properties of expectation and variance.

	
		\begin{thm}
			If $y$ is $n \times 1$ random vector, with mean vector $\mu$ and covariance matrix $V$, and $A$ is a matrix of constants, then
			$$E(y'Ay) = \tr(AV) + \mu'A \mu$$
		\end{thm}
	
		\begin{proof}
			 We multiply and use linearity of expectation, expansion of covariance
			\begin{align*}
				E(RSS) &= E[Y'AY] = E\BB{ \sum_i^n \sum_j^n a_{i,j} y_i y_j } =   \sum_i^n \sum_j^n a_{i,j} E\BB{ y_iy_j }
				\intertext{Expanding with covariance, and with $(\sigma_{i,j}) = Cov(Y) = V$}
				&= \sum_i^n \sum_j^n a_{i,j} \BB{\Cov(y_i, y_j) + E(y_i)E(y_j)}
				= \sum_i^n \sum_j^n a_{i,j} \sigma_{i,j}  + \sum_i^n \sum_j^n a_{i,j}  \mu_i \mu_j  \\
				&= \tr(A V ) + \mu' A \mu			
			\end{align*}
		\end{proof}
	
		\begin{proposition}
			$E(RSS) = (n-p-1)\sigma^2 + \mu' A \mu$ where $A = I-H$
		\end{proposition}
	
		\begin{proof}
		Using the above,
			Set $A = I-H, V = \sigma^2 I$, then 
				$$\tr(AV) = \tr[(I-H) \sigma^2 I] = \sigma^2 \tr(I-H)$$
			Expanding, $ \tr(I-H) = \tr(I_n) - \tr(H) = n - p - 1$; where $\tr(H) = \tr(X(X'X)^{-1}X') = \tr(X'X(X'X)^{-1}) = \tr I_{p+1} = p+1$ since $(X'X)^{-1}$ is $(p+1) \times (p+1)$.
		\end{proof}
		{\bf This will be on the final!}
		\begin{proposition}
			 $\mu' A \mu = 0$, where $A = I-H$, $\mu = X \beta$.
		\end{proposition}
		\begin{proof}
			\begin{align*}
				\mu' A \mu &= (X\beta)'(I - X(X'X)^{-1}X')X\beta = \beta'X'X\beta - \beta'X'X(X'X)^{-1}X'X\beta\\ 
				&= \beta'X'X\beta - \beta'X'X'\beta \\
				&= 0
			\end{align*}
		\end{proof}
		
		\begin{proposition}
			$E(RSS) = (n-p-1)\sigma^2$
		\end{proposition}
	
		This follows from substitution into the past 3 statements. The following proposition also easily follows.
		
		\begin{proposition}
			$E(MRSS) = E(\f{RSS}{n-p-1}) = \sigma^2$
		\end{proposition}
		
		\subsubsection{$RSS$ and $SS_{reg}$ for Multiple LR} 
		
		By Gauss-Markov assumptions, $\eps_i \sim N(0, \sigma^2)$, and so $\f{\eps_i}{\sigma} \sim N(0,1)$ by $Z$-score. Also this gives $\f{1}{\sigma} \eps \sim N(0, I)$. Note	\footnote{The slides use $Q = I - H$, but we use $A$ as before.}	
		$$e = Y - X \hat\beta = Y - HY = AY$$ 
		Our underlying model is assumed to be $Y = X\beta + \eps$, so therefore $Ay = AX\beta + A\eps$. Expanding and since $HX = X$, 
		$
		AX\beta = (I - H) X \beta = 0
		$ 
		so $e = Ay = A \eps$. That is our observed errors are the difference $\eps - H\eps$; the error vector minus its projection. This proves the following fact
		
		\begin{fact}
			$e = (I - H) \eps$.
		\end{fact}
		
		We also showed $A = I-H$ is {\bf symmetric and idempotent}; this implies 
			$$A'A = A^2 = A$$
		Then 
		$$
			RSS = (y - \hat y)'(y - \hat y) =  e'e = \eps' A' A \eps = \eps A \eps  = \sigma^2 Z'AZ
		$$
		This implies $\df{RSS}{\sigma^2} = Z'AZ$.
		
		\begin{thm}
			If $A$ is a symmetric and idempotent $n \times n$ matrix and $Z \sim N(0, I)$, then $Z'AZ \sim \chi^2(\tr(A))$
		\end{thm}
			
		No proof, try it yourself for practice. However, notice $Z'Z \sim \chi^2(n)$ and use a nice basis for a projection operator. Recall $A = I-H$ so this gives 
		$$\df{RSS}{\sigma^2} \sim \chi^2(\tr(A)) = \chi^2(n-p-1)$$
	
		\begin{proposition}
			$\ol y = (1'1)^{-1} 1' y$.
		\end{proposition}
		
		Therefore we may rewrite the regression sum of squares involving $y$ and $H$.
		
		\begin{proposition}
			$SS_{reg} = y' [H - 1(1'1)^{-1}1']y$
		\end{proposition}
		\begin{proof}
		First, write $$
			SS_{reg} = [\hat y - 1 \ol y]' [\hat y - 1 \ol y] = y' [H - 1(1'1)^{-1}1]'[H - 1(1'1)^{-1}1']y
		$$
		Now we show $[H - 1(1'1)^{-1}1]'[H - 1(1'1)^{-1}1'] = H - 1(1'1)^{-1}1'$. Expanding,
		\begin{align*}
			[H - 1(1'1)^{-1}1']'[H - 1(1'1)^{-1}1'] 
			&= H^2 - 1(1'1)^{-1}1'H - H 1(1'1)^{-1}1' + 1(1'1)^{-1}1' 1(1'1)^{-1}1'\\
			\intertext{Note since $HX = X$ and $1$ is a column in $X$, then $H\cdot \mathbf 1 = \mathbf 1$, and taking transposes a similar result holds.}
			&= H - 1(1'1)^{-1}1' - 1 (1'1)^{-1}1' + 1(1'1)^{-1}1' \\
			&= H - 1 (1'1)^{-1}1'
		\end{align*}
		\end{proof}
		This gives us a way to express the regression sum squared using $y, H$ and a constant matrix. We use this to show that the regression sum squared, and residual sum squared from above are independent.
		
		\begin{proposition}
			The regression sum of squares $SS_{reg} = y' [H - 1(1'1)^{-1}1']y$ and residual sum of squares $RSS = y' [ I - H] y$ are {\bf independent}.
		\end{proposition}
		
		We do this by computing $[H - 1(1'1)^{-1}1'] \sigma^2 I [I - X(X'X)^{-1}X'] = \sigma^2(H-B)(I-H) = 0$ as an exercise.
		
		\subsubsection{ANOVA Table for MLR}
		
		
		\begin{table}[ht]
			\centering
			\begin{tabular}{lrrrrr}
				\hline
				Sources of Variation & Df & Sum Sq & Mean Sq & F value & Pr($>$F) \\ 
				\hline
				Regression & $p$ & $SS_{reg}$ & $MS_{reg} = \f{SS_{reg}}{p}$  & $F_0 = \f{MS_{reg}}{MRSS}$ & etc \\ 
				Residuals & $n-p-1$ & $RSS$ & $MRSS_{reg} = \f{RSS}{n-p-1}$ &  &  \\ 
				\hline
				Total & $n-1$ & SST &&&
			\end{tabular}
		\end{table}
	
		By independence, and since $SS_{reg} \sim \chi^2(p)$, $RSS \sim \chi^2(n-p-1)$, then we have 
		$$\f{SS_{reg}/p}{RSS/n-p-1} \sim F(p, n-p-1)$$
		We may perform an $F$ test with the null hypothesis $$H_0: \beta_0 = \beta_1 = \ldots = \beta_p = 0 \text{ and } H_1: \beta_i \neq 0 \text{ for some } i$$
		Significance in the statistic gives evidence for at least one predictor being valid; at least some $X_i$ explains a significant proportion of the variance in $Y$.\\
		
		Recall that the coefficient of determination $R^2 = \f{SS_{reg}}{SST}$. As the number of variables increases, so does $R^2$, since more predictors decrease $RSS$. 
		$$
			RSS = \sum (y_i - \beta_0 - \beta_1 x_i - \ldots - \beta_px_p)^2
		$$ 
		where an additional predictor will decrease each term in the sum. Note when we have $n$ predictors for the sample size, we have a perfect fit and $R^2 = 1$. Geometrically, projection plane induced by $H = X(X'X)^{-1}X'$ is the whole space. In short, we get many predictors but none of them good and we overfit. We account for the number of predictors using an adjusted $R^2$
			$$
				R^2_{adj}  = 1 - \f{RSS / n-p-1}{SST / n-1}
			$$
		The interpretation is exactly the same, but is a more robust statistic in multiple linear regression due to the previous issues.
		
		\subsection{Partial F-test}
		
		One of the most important tests in an MLR is the partial F-test. In ANOVA we do a test for the full model; we identify whether there is any significant predictor. The {\bf partial F-test} identifies whether a subset of predictors still significantly predicts the response. However, the {\bf null hypothesis is that the reduced model is better} than the full model. Remember that we consider the ratio of the error sum squared; a significant increase in errors after removing predictors indicates a worse model, and larger $F$-statistic.\\
		
		Suppose we have two models. The {\bf full} $Y = \beta_0 + \beta_1 X_1 + \ldots + \beta_p X_p + \eps$. We test whether the model still explains the response when we remove the first $k$ predictors; we consider the {\bf reduced} $Y = \beta_{k+1} X_{k+1} + \ldots + \beta_p X_p + \eps$. First, write 
			$$RSS(\text{reduced}) - RSS(\text{full}) = y'[H - H_1]y$$ 
		Without proof, but similar to for $RSS$ before, 
		$$RSS(\beta_2 \mid \beta_1)/\sigma^2 = (RSS(\text{reduced}) - RSS(\text{full})) / \sigma^2 \sim \chi^2(k)$$
		Thus 
		$$\df{RSS(\beta_2 \mid \beta_1)/\sigma^2}{RSS(\text{full}) / n-p-1} \sim F(k, n-p-1)$$
		We test
		$$
			H_0 : \text{ reduced model is better fit}, \quad H_1: \text{ full model is better fit }
		$$
		A large $F$ value suggests that the reduced model explains much less variability than the full model, and fits the data worse. This implies we should be rejecting the null, so predictors cannot be removed from the model. Small values imply that both reduced and full models explain a similar amount of variability, so the additional predictors may not be necessary.\\
		
		Opposite test hypotheses occur, since we test ratios of {\bf residuals}; high ratio means large residuals in reduced model.
		
		\subsection{Diagnostic checking}
		
		The three assumptions of linear regression are (1) linearity, (2) homoscedasticity, (3) independence of the errors, with normality also being one. One of the most important tasks is {\bf checking the assumptions} in our data. This is called diagnostic checking. Anscombe's datasets give an example of why checking these assumptions is important; the models give the same predictors but differ greatly in their structure. \\
		
		Suppose we fit $Y = \beta_0 + \beta_1 X + \eps$. The fitted regression $\hat y = \hat \beta_0 + \hat \beta_1 X$ produces the estimate for $E(Y \mid X)$. $e$ is an unbiased estimate for $\eps$. A good way to check is to plot the residuals, there should be no pattern and should be a random scatter plot. We can also plot residuals against $\hat y$ as in multiple LR. Assumptions hold if there is {\bf no pattern}. Other relationships, like a quadratic one, will become apparent in the residuals. The following steps are best practice:
		\begin{enumerate}
			\item Assess model assumptions using residual plot. There should be no pattern.
			\item Determine which data points have $x$-values with large effect on $Y$. ({\bf Leverage points.})
			\item Determine which points are outliers in their responses.
			\item Assess the influence of bad leverage points on the fitted model.
			\item Examine whether constant error variance assumption is reasonable. (Do residuals vary with $X$?)
			\item If data is collected over prolonged period of time, see if it is corelated with time.
			\item For small sample size or prediction intervals, assess whether normality of errors is reasonable. (Normality tests?)
		\end{enumerate}
	
	 If this is successful, then our assumptions are valid and our predictors can be trusted. If the assumptions fail, our analysis is invalid.
		
	\section{June 1: Lecture 6}
	
	\subsection{Leverage Points}
	
	{\bf Leverage points} are observations that are highly influential on the fitted regression line. Leverage points occur due to an a value of $X=x$ far from $\ol x$. The corresponding $Y=y$ greatly influences the line for a given $X=x$. Such a pair $x,y$ that greatly changes the least square estimates is a {\bf bad} leverage point. For extreme $x$, if $y$ is close to the fitted line it is a good leverage point, but if it is far it is a bad one. An {\bf outlier} is an observation that takes an extreme $y$ value for an $x$ that is not far from $\ol x$. 
	
	\subsubsection{Numerical Summary}
	Recall
	$$
		\hat\beta_0 = \ol y - \hat \beta_1 \ol x \qquad \hat\beta_1 = \sum_j^n \f{x_j - \ol x}{S_{X,X}} y_j = \sum_j^n c_jy_j
	$$
	Then
	\begin{align*}
		\hat y_i &= \hat \beta_0 + \hat \beta_1 x_i = \ol y + \hat \beta_1 (x_i - \ol x) = \sum_j^n \BB{\f{y_j}{n} + \f{(x_j - \ol x) (x_i - \ol x)}{S_{X,X}} y_j} \\
		&= \sum_j^n \BB{\f{1}{n} + \f{(x_j - \ol x) (x_i - \ol x)}{S_{X,X}} } y_j = \sum_j^n h_{i,j} y_j
	\end{align*}
	This $h_{i,j}$ is the entry in the hat matrix $H$. When $i = j$, then $\displaystyle h_{i,i} = \f{1}{n} + \f{ (x_i - \ol x)^2}{S_{X,X}}$. We show $\sum_j^n h_{i,j} = 1$. Further, we can write $\hat y_i = h_{i,i} y_i + \sum_{j \neq i} h_{i,j}y_j$. If we have $h_{i,i} \approx 1$, then $\hat y_i$ is close to $y_i$, and it is a leverage point. It can also be shown $\text{mean}(h_{i,i}) = \f2n$ (by definition). Using this, a popular way to identify a leverage point is to check if $h_{i,i} > \f4n$, or twice the mean. This is a useful rule of thumb.\\
	
	Leverage is concerned with a single observation far from the rest of the data in the $x$-space. We have two ways of dealing with bad leverage points. We can (1) remove the data point or (2) fit a different regression model. A quadratic or logarithmic transformation of $X$ may be needed.
	
	\subsection{Standardized Residuals and Influential Points}
	
	In the real world, often people work in {\bf sensitivity analysis}. This is essentially identifying influential points, which we discuss.\\
	
	Residuals reflect the difference between observed and predicted response. We might want to use them to measure the influence a leverage point will have on the estimated line. It turns out that the estimated residuals do not always have the same variance; $V(e_i)$ is not the same for all $i$. Actually, we find $V(e_i) = \sigma^2(1-h_{i,i})$. We prove this.
	\begin{proposition}
		$\sum_{j=1}^n h_{i,j}^2 = h_{i,i}$
	\end{proposition}
	\begin{proof}
		\begin{align*}
			\sum_{j=1}^n h_{i,j}^2 
			&= \sum_{j=1}^n \BB{ \f1n +  \f{(x_j - \ol x) (x_i - \ol x)}{S_{X,X}} }^2 = \sum_{j=1}^n \BB{\f1{n^2} + \f2n \f{(x_j - \ol x) (x_i - \ol x)}{S_{X,X}} + \f{(x_j - \ol x)^2 (x_i - \ol x)^2}{S_{X,X}^2}} \\
			&= \f1{n} + \f2n \f{\sum_{j=1}^n(x_j - \ol x) (x_i - \ol x)}{S_{X,X}} + \f{\sum_{j=1}^n(x_j - \ol x)^2 (x_i - \ol x)^2}{S_{X,X}^2}  \\
			&= \f1n + \f{ (x_i - \ol x)^2}{S_{X,X}} = h_{i,i}
		\end{align*}
	\end{proof}
	We use this in the last steps to prove the following.
	\begin{proposition}
		\label{e_var}
		$V(e_i) = \sigma^2(1-h_{i,i})$
	\end{proposition}
	\begin{proof}
			\begin{align*}
				V(e_i) &= V\BB{y_i(1-h_{i,i}) + \sum h_{i.j}y_j} = (1-h_{i,i})^2V(y_i) + \sum_{i \neq j} h_{i,j}^2 V(y_j) \\
				&= \sigma^2 \BB{ (1-h_{i,i})^2 + \sum_{i \neq j} h_{i,j}^2 } = \sigma^2 \BB{ (1-h_{i,i})^2 + \sum_{i}^n h_{i,j}^2 - h_{i,i}^2} \\
				&= \sigma^2 \BB{ 1 - 2 h_{i,i} + h_{i,i}^2 + h_{i,i} - h_{i,i}^2} =  \sigma^2(1-h_{i,i})
			\end{align*}
	\end{proof}
	We can now discuss the variation in each residual using our hat matrix. We see that the estimated residuals are not actually independent, event though we assume that the errors are. If $e_i$ were independent, we would expect $Var(e_i) = \sigma^2$. However, we have an extra term of $-\sigma^2 h_{i,i}$, which indicates the variance of a residual depends on its distance from $\ol x$. Residuals are correlated, but the correlation is small.\\
	
	This makes it difficult to know whether the patterns we see are due to model violations or variance of the residuals. To overcome this issue, we {\bf standardize} the residuals by dividing by their {\bf standard error}. By prop. \ref{e_var}, 
	$$
		\text{se}(e_i) = s\sqrt{1 - h_{i,i}} \imp r_i = \f{e_i}{s\sqrt{1 - h_{i,i}}}
	$$
	Where  $s^2 = \df{\sum e_i^2}{n-2}$. Note $r_i \sim t(n-2)$, so these are also called `studentized' residuals. If high leverage points exist, it is more important to look at plots of standardized residuals; we can just check if $r_i \in [-2,2]$ or $[-4,4]$. It is expected that the variance of $r_i$ will be larger for center values of $X$, and smaller for remote values. Then looking at the plot, we can identify whether a residual corresponds to an outlier; we plot standardizes residual against dependent variable.
	
	\begin{example}
		In our Treasury Bond example, we identify three bad leverage points by plotting studentized residual against dependent variable. Viewing these in detail, we find that they are `flower bonds', so we remove them from the analysis. The remaining points are more or less linear, but a slight bend may give evidence that it is a logarithmic relationship.
	\end{example}

	\subsection{Cook's Distance}
	
	How can we quantify the influence a small number of observations on the regression line with a single statistic? In 1977, Cook provided the following expression to calculate the influence of a single point on the regression line.
	
	\begin{definition}
		The {\bf Cook's distance} for $(x_i, y_i)$ is given by
		$$
		D_i = \f{(\hat y_{j (i)} - \hat y_j)^2}{2s^2} = \f{r_i^2}{2} \cdot \f{h_i}{1-h_i}
		$$
		where the subscript $i$ references the predicted value from a model fit without $(x_i, y_i)$. Thus $\hat y_{j (i)}$ denotes the $j$th fitted value based on the fit when the $i$th observation is deleted from the fit.
	\end{definition}

	A high Cook's distance means the model is a  {\bf bad fit} for the $i$-th observation, since there is a large residual or it sits far from the centre of the predictors.	There are similar metrics in MLR which we discuss later. The second expression is easier to work with since it does not require refitting of any models. Large Cook's distance means large $r_i$ or large $h_{i,i}$. We use the cutoff $D_i > \df4{n-2}$ as a rough cutoff guideline, but identifying unusual $D_i$ is most important.
	
	\begin{example}
		In the previous Treasury Bond example, the 3 unusual observations have a very high Cook's distance when plotted, and are valid to be removed.
	\end{example}

	\subsection{Normality of the Errors}
	
	We need to assume $\eps_i$ is normally distributed to perform $F$, $t$, and $Z$ tests, as well as construct confidence intervals. We will verify the normality assumption using residual plots. First, we can show $\sum h_{i,j} = 1$ and $\sum x_jh_{i,j} = x_i$,
	\begin{proposition}
		$e_i = \eps_i - \sum_j^n h_{i,j} \eps_j$
	\end{proposition}
 	\begin{proof}
 		$$
 		e_i = y_i - \hat y_i = y_i - \sum_j^n h_{i,j}y_j = 
 		\beta_0 + \beta_1 x_i + \eps_i - 
 		\sum_j^n h_{i,j}(\beta_0 + \beta_1 x_j + \eps_j) 
 		= \eps_i - \sum_j^n h_{i,j} \eps_j
 		$$
 	\end{proof}
	In small sample sizes, the second term may dominate, and the residuals may look normal even if the $\eps_i$ are not.
	As $n$ increases, the second term in the last equation has a smaller variance than the first term, so the first term dominates the last equation. For large samples, the residuals can be used to assess normality of the samples.\\
	
  	A common way to assess normality is via a {\bf QQ-plot}; the studentized errors are plotted against their quantiles. If the quantiles match that of a normal distribution, the plot is close to the $y=x$ line, and the normality assumption is valid. We must also check that the constant variance assumption is met; we cannot use inferential tools if it is not true. 
  	
  	\subsection{Variance stabilizing transformations}
  	
  	In the slides example, constant variances is violated. For inference, our prediction intervals depend on $X$. A transformation of $Y$ can stabilize the variance: make it not depend on $X$.\\
  	
  	When we are counting events, as in the Slide 28 example, we typically fit a Poisson distribution. In a Poisson distribution, the mean and variance are both $\lambda$. Since in regression we model the conditional mean $E(Y \mid X) = \lambda_X$, we have also a conditional variance: $\lambda_X$ changes by $X$, so should the variance. The {\bf square root transformation} can help in this situation.\\
  	
  	Taking the function of a random variable, $f(Y) \approx f(E(Y)) + f'(E(Y))(Y-E(Y))$. Taking the variance, we get
  	$$
  		V(f(Y)) = \BB{f'(E(Y)}^2V(Y)
  	$$
  	since $E(Y)$ is a constant, and using variance properties. This way of approximating variance is called the {\bf delta method}. In the Poisson example, $E(Y \mid X) = V(Y \mid X) = \lambda(x)$. Letting $f(Y) = \sqrt(Y)$, then
  	$$
  		V(Y^{0.5} \mid X) = \BB{0.5 E(Y \mid X)^{-0.5}}^2V(Y \mid X) = \BB{\f12}^2 \lambda(x)^{-1} \lambda(x) = \f14
  	$$
  	which makes $V(f(Y) \mid X)$ constant. In the example, $X,Y$ are both counts, so we perform the square root transformation on both and keep the same units. The variance stabilizing transformation stabilizes prediction error across the predictor variable. Our predictions may vary, but we keep the transformed model. We may not always get count data, so depending on the relationship between variance and mean we might use different transformations:
  	
  	\begin{center}
  		\begin{tabular}{ c c }
  			\hline
  			Relationship & Transformation \\
  			\hline
  			$\sigma^2 \propto E(Y)(1-E(Y))$ & $y^* = \sin^{-1}(\sqrt{y})$ \\
  			$\sigma^2 \propto E(Y)^2$ & $y^* = \log y$ \\ 
  			$\sigma^2 \propto E(Y)^3$ & $y^* = y^{-\f12}$ \\
  			$\sigma^2 \propto E(Y)^4$ & $y^* = y^{-1}$ \\
  			\hline
  		\end{tabular}
  	\end{center}
  	
  	We can verify that the delta method is variance stabilizing. We need to make sure that interpretability is not lost: in practice a transformation is chosen empirically. There is no exact rule about which transformation is best for a set of data. Transformations of $X$ are discussed later.
  	
  	\section{June 6: Lecture 7}
  	
  	\subsubsection{Assignment 2 Instructions}
  	The idea is to create a regression model, and defend validity of the model using concepts learned in class. We use the NHANES dataset, including demographic information. We do both inference and prediction using this regression model; create training and test sets. We create a {\bf cross sectional} dataset, where each individual is considered independent. We will elaborate about the theory next Monday. \\
  	
  	Word limit 1000 excluding captions and figures. Maximum 5 tables and figures. Up to 3 additional tables and figure should be included in an appendix if they are relevant to the analysis. {\bf Due June 18.}
  	
  	\subsection{Transformation for Non-Linearity}
  	
  	The final thing we discuss in simple linear regression are transformations for non-linearity. We have seen these before in variance stabilizations. These transformations are also applied when there is non-linearity so that we get some linear relationship after transformation. For example, consider the true model
  	$$
  		Y = \beta_0 X^{\beta_1}
  	$$
  	
  	Then transforming $Y^\star = \log(Y)$ and $X^\star = \log(X)$ (natural log) then
  	$$
  		Y^\star = \log \beta_0 + \beta_1 X^\star
  	$$
  	Then $Y^\star$ is linear with the new transformed $X^\star$. We can now use least squares to fit a relationship between $Y^\star, X^\star$, and recover $\beta_0$ with $\exp$.
  	
  	\begin{example}
  		In slide example, maximum salary regressed on score is a good linear fit, but the standardized residuals show a quadratic curve-like relationship; there is an assumption-violating pattern. Assuming the underlying model is $Y = \beta_0 X^{\beta_1}$, we fit a linear relationship, and find that the new standardized residuals show some pattern.
  	\end{example}
  	
  	\subsection{The Box-Cox Transformation}
  	To remain interpretable, the scaling of $X, Y$ must be the same. In order to best get rid of non-linearity, we use the {\bf Box-Cox transformation}.\\
  	
  	We have often seen some kind of {\bf power transformation} on $Y$:
  	$$
  		\psi(y,\lambda) = y^\lambda 
  	$$ 
  	instead of $y$. To determine the most appropriate value of $\lam$, we use {\bf maximum likelihood estimation}. We assumed $Y = \beta_0 + \beta_1X + \eps,\, \eps \sim N(0, \sigma^2)$ so that 
  	$$
  		Y \mid X \sim N(\beta_0 + \beta_1X, \sigma^2)
  	$$
  	Therefore 
  	$$
  		\mathcal{L}(\beta_0, \beta_1, \sigma^2) = \prod_{i}^{n} \f{1}{\sqrt{2\pi\sigma^2}} \exp\BB{-\f12 \f{(y_i - \beta_0 - \beta_1 x_i)^2}{\sigma^2}} = \BB{2\pi\sigma^2}^{-\f{n}2} 
  		\exp\BB{-\f1{2 \sigma^2 } RSS}
  	$$
  	Therefore maximum likelihood estimate is the same as when $RSS$ is minimized; the estimate for $\beta_0, \beta_1$ we first developed. For the Box-Cox transformation, we fit the model parameters $\beta_0, \beta_1$ to transformed $RSS$: 
  	$$RSS = \sum_i^n (\psi(y_i, \lam) - \beta_0 - \beta_1 x_i)^2$$ 
  	For this expression, we minimize the fitted $RSS$ over all possible $\lam$ numerically; since we cannot do so analytically. In other words, for some $\lam$, fit $\beta_0, \beta_1$ so that $RSS$ is minimized, and take this minimum over all possible $\lam$. Problems arise when $\lam = 0$, where the response becomes constant. We therefore use $\psi(y, \lambda) = \f{y^\lambda - 1 }\lam$, since $\lim_{\lam \to 0}\f{y^\lambda - 1 }\lam = \log y $. However, small change of $\lam$ greatly changes $\psi$, so we set
  	$$
  		\psi(y, \lambda) = \begin{cases}
  			\df{\text{gm}(Y)^{\lam - 1}y^\lambda - 1 }\lam & \lam \neq 0 \\
  			\text{gm}(Y)\log(Y) & \lam = 0
  		\end{cases}
  	$$
  	where $gm(Y) = \exp \BB{ \f1n \sum_i^n \log(Y_i)}$.
  	This is the {\bf Box-Cox transformation}. Adding geometric mean is not always necessary. We can also tranform the predictor variable:
  	$$
  		\psi(X, \lambda) = 
  		\begin{cases}
  			\df{X^\lambda - 1 }\lam & \lam \neq 0 \\
  			\log(X) & \lam = 0
  		\end{cases}
  	$$
  	That is, fit $E(Y \mid X) = \alpha_0 + \alpha_1 \psi(X, \lambda)$, and find maximum of maximized MLE for all possible $\lam$.
  	Note we do not multiply by G.M., since we do not need to stabilize $X$. We now have $\psi(Y, \lambda_Y), \psi(X, \lambda_X)$ where we maximized MLE for these values of $\lambda_y$, $\lambda_X$. We can replace {\bf both} $X,Y$ with $\psi(Y, \lambda_Y), \psi(X, \lambda_X)$, and maximize, to choose the best transformation. 	This is a nightmare for interpretation though. In the example, $\lambda_Y = 0$, $\lambda_X = 0.5$ seems to create the best fit. \\
  	
  	Although these transformations are terrible for interpretability, they increase the predictive power of the model. The problem of interpretability vs. predictability is a major one in data science. In a predictive model, we use these transformations since they help correct modelling assumptions and improve predictive power. Usually log or square root transformations correct a skew in either variable, and the choice depends on the data. 
  	
  	
  	
  	
  	
  	
  	\subsection{Diagnostics in Multiple Linear Regression}
  	
  	Checking the model assumptions is actually simpler in MLR than in SLR.
  	
  	\subsubsection{Leverage Points}
  	
  	A {\bf leverage point} is one that lies far from the rest of the observations with respect to its predictor values. The least squares procedure fits a plane that minimizes the distance between each point and this plane. While it does not mean that a leverage point will be influential to the model fit, this potential to influence the line is why we identify these points.
  	
  	Recall that the projection of $Y$ onto $X$, 
  		$$
  			\hat Y = X \hat \beta = X(X'X)^{-1}X' Y = HY
  		$$
  	
  	where $H$ is an $n \times n$ matrix with rank $p+1$, where $p+1$ is the number of $\beta_i$. We denote $H = (h_{i,j})_{1 \leq i,j \leq n}$. Then 
  	$$\hat Y = HY \imp \hat y_i = \sum_{j}^n = h_{i,j}y_j $$
  	If an observation is a leverage point, the fitted value is strongly attracted to the observed value. We concern ourselves with the diagonal elements of the hat matrix $h_{i,i}$. Unlike simple LR, $h_{i,j}$ are not easy to calculate, so we rely on software for the hat matrix. An observation is a leverage point if $h_{i,i} > 2 \f{p+1}{n}$.
  	
  	\begin{example}
  		\verb*|model.full| in R
  	\end{example}
  
  	\subsubsection{Standardized Residuals}
  	
  	Recall $e = (I-H)Y \imp e_i = (1-h_{i,i}) y_i - \sum_{i\neq j} h_{i,j} y_j$. We can standardize the residuals similar to SLR. We can show $V(e_i) = (1-h_{i,i}) \sigma^2$, so it is best to standardize when they have constant variance. We do this by
  	$$
  		r_i = \f{e_i}{s\sqrt{1- h_{i,i} }}
  	$$ 
  	where $s$ comes from the MLR version, $s = \sqrt{\df{RSS}{n-p-1}}$.
  	Like in SLR, these can be used to detect outliers and QQ-plot to test normality assumptions. However, it is difficult to test their relationship with the predictors since there are many of them, so plots against individual predictors are used. Any pattern shows that assumptions are violated.
  	
  	\subsubsection{Influential Observations}
  	
  	We saw already that we need to be concerned with leverage points and outliers. If both of these observations have the potential to influence the regression line, then we need a way to determine which observations we should be concerned with. Such observations are {\bf influential observation} for the regression line. We quantify the amount of influence each observation has in three ways.
  	
  	\begin{definition}
  		In MLR, the {\bf Cook's distance} is
  		$$
  			D_i = \f{(\hat Y_{(i)} - \hat Y )'(\hat Y_{(i)} - \hat Y )}{(p+1)S^2} = \BB{\f{r_i^2}{p+1}} \cdot \BB{
  					\f{h_{i,i}}{1-h_{i,i}}}
  		$$
  	\end{definition}
  	A point can be an {\bf influential observation} if the model fits the $i$-th observation poorly, giving a large Cook's Distance. While the Cook's distance looks at the effect of a single observation on all fitted values, we can quantify the effect on its own fitted value. This is quantified with 
  	\begin{definition}
  		The {\bf DFFITS statistic} 
  		$$
  			DFFITS_i = \f{y_i - \hat y_{i(i)}}{\sqrt{S^2_{(i)} h_{i,i}}}
  				= \BB{\f{h_{i,i}}{1-h_{i,i}}}^{\f12}\f{e_i}{s_{(i)} \sqrt{1-h_{i,i}}}
  		$$
  		where $\hat y_{i(i)}$ is the predicted value for the observation $i$ if it {\bf not} included in the model.
  	\end{definition}
  	If the residual with the observation removed is very large, then it does not lie close to the fitted regression. The equivalent expression looks similar to the Cook's distance, but does not provide many advantages compared to Cook's distance. Cook's distance is more important. With DFFITS, an observation is considered influential if $\abs{\text{DFFITS}_i} > 2 \sqrt{\df{p+1}{n}}$.\\
  	
  	Another statistic for identifying influential points if the  {\bf DF BETAS}. It directly quantifies the effect of the $i$-th observation on the least squares 
	\begin{definition}
		The {\bf DF BETAS} are calculated as
		$$\text{DFBETAS}_i = \f{\hat\beta_j - \hat \beta_{j(i)}}{\sqrt{S^2_{(i)}) (X'X)^{-1}_{j,j}}}$$ Here $\beta_{j(i)}$ is the estimated coefficient for predictor $j$ when $i$ is not included in the data. This statistic is calculated for all $n$ observations. A large change in the predictors when observation $i$ is removed means the observation greatly influences the fit of the regression line.
		Typically the $i$-th observation is influential if $\abs{\text{DFBETAS}_i} > \df2{\sqrt{n}}$.
	\end{definition}
	
	All of the above statistics may give different significant observations, but we should not disregard any of them.
	
	\subsubsection{Non-Linearity}
	
	We have again assumed the relationship is linear. If the true relationship is non-linear: $E(Y \mid X) = g(\beta_0 + \beta_1 X_1 + \ldots + \beta_p X_p)$ then we still use Box-Cox to transform $X,Y$. We can transform the response $Y$, or transform both. To transform $Y$, we still use 
	
	$$
	\psi(y, \lambda) = \begin{cases}
		\df{\text{gm}(Y)^{\lam - 1}y^\lambda - 1 }\lam & \lam \neq 0 \\
		\text{gm}(Y)\log(Y) & \lam = 0
	\end{cases}
	$$
	where $\lam$ is chose by maximizing the MLE where $y_i$ is replaced with $\psi(y_i, \lam)$.
	
	Summary: in diagnostics we have leverage points or outliers, we calculate cooks distance, DFFITS, if there is nonlinearity we can choose a transformation according to box-cox.
	
	\subsection{Corelated Predictors}
	
	{\bf In sum:} What if $X'X$ is not invertible? When does this occur, and what do we do?\\
	
	In Task 2 of A1, we saw that fitting an SLR to corelated predictors lead to biased sampling distributions of the predictor with smaller variance. Total corelation of $X_i, X_j$ leads to linear dependence of columns in $X'X$, making it non-invertible, and so we cannot fit a model. \\
	
	When predictors are corelated, then that affects their individual relationship with the outcome. Predictors could be weakly, moderately, or strongly correlated. If $\Corr \approx 1$, we cannot obtain a least squares estimate. Even with moderate correlation, we might still have to be careful, since multicolinearity and non-full rank matrix may occur. This affects prediction etc. \\
	
	When $X = [X_1, \ldots, X_p]$ is the covariance matrix, if for some $t_i$, $\sum t_j X_j = 0$ the columns are linearly dependent; $(X'X)^{-1}$ is not invertible. But if corelations between predictors are very high, then $\det(X'X)$ will be close to $0$ and issues may occur.\\
	
	Assuming we have a linear model $y = \beta_1 x_1 + \beta_2 x_2 + \eps$, it is not difficult to see
	
	\begin{example}
		Assuming we have a linear model $y = \beta_1 x_1 + \beta_2 x_2 + \eps$, we see
		$$X'X \hat \beta = X'y \imp \pmat{1 & r_{12} \\ r_{21} & 1} \pmat{\hat \beta_1 \\ \hat \beta_2} = \pmat{r_{1y} \\ r_{2y}}$$ where $r_{12} = \Corr(X_1, X_2)$ and $r_{j,y} = \Corr(X_j, Y)$. Thus $\det X'X = 1 - r_{12}^2$. As $r_{12} \to 1$ then this determinant gets small, and $X'X$ becomes singular. Moreover, $V(\hat \beta) \to \infty$ since $V(\hat \beta \mid X) = (X'X)^{-1}\sigma^2$. So for high $r_{12}$, our confidence intervals become very wide and unreliable.
	\end{example}

	We cannot remove the extra linearly dependent variable; as we saw in the midterm, this creates bias in the other predictor. Let's assume $C = ({X'X})^{-1}$ and $V(\hat \beta_j \mid X) = \sigma^2 C_{j,j}$. When we have $>2$ predictors, it can be shown that 
	$$C_{j,j} = \f{1}{1-R_j^2}$$ 
	where $R_j^2$ is the coefficient of multiple determination of $X_j \sim X_1 \ldots X_n$.  $C_{j,j}$ is the {\bf variance inflation factor}. The first thing we check is $\text{VIF} > 5$, if so we deal with such variables separately or at least address them.
	
	\section{June 8: Lecture 8}
	
	\subsection{Handling Multicolinearity}
	
	To handle multicolinearity we can either collect more data, or re-specify the model. By removing one of the correlated predictors, the effect of multicolinearity should be reduced. However, if the wrong predictor is removed, then it may reduce the predictability of the model. 
	
	\subsection{ANCOVA: Analysis of Covariance}
	
	We discussed dummy variables; if $X = 0,1$ then 
	$$E(Y \mid X) = \beta_0 + \beta_1 X \imp E(Y \mid X = 0) = \beta_0,\quad E(Y \mid X = 1) = \beta_0 + \beta_1$$
	What if we have multiple categorical predictors (age, sex, etc.)? Then we create multiple categorical predictors, $X_1, X_2$ and fit the MLR model $E(Y \mid X) = \beta_0 + \beta_1 X_1 + \beta_2 X_2$. As a specific case, 
	$$E(Y \mid X_1 = 0, X_2 = 1) = \beta_0 + \beta_2$$
	In order to view the significance of each categorical predictor, we do ANOVA.When $X_1$ is categorical, $X_2$ is continuous and we fit an MLR model, then 
	$$E(Y \mid X_1 = 0, X_2) = \beta_0 + \beta_2X_2,\qquad E(Y \mid X_1 = 1, X_2) = \beta_0 + \beta_1 + \beta_2X_2$$
	We get two lines with the same slope, but the intercept changes with the categorical $X_1$. Often $X_2$ is referred to as the {\bf effect}. However, given a change in the categorical predictor, we may expect a more rapid increase in $X_2$. I.e. smoking may cause blood pressure to increase more rapidly with age. Then we have 
	$$E(Y \mid X_1 = 0, X_2) = \beta_0 + \beta_2X_2,\qquad E(Y \mid X_1 = 1, X_2) = (\beta_0 + \beta_1) + (\underbrace{\beta_{1,2}}_{\text{interaction effect}} + \beta_2)X_2$$
	$\beta_{1,2}$ is often called the {\bf interaction effect}, or the {\bf difference in difference} parameter, while the parameters $\beta_1, \beta_2$ are the {\bf main effects}. Our underlying model is 
	$$E(Y \mid X_1, X_2) = \beta_0 + \beta_1 X_1 + \beta_2X_2 + \underbrace{\beta_{1,2}X_1X_2}_{\text{interaction}}$$
	the regression lines are no longer parallel. Slopes should be interpreted separately for the categorical $X_1$.
	
	\begin{example}
		We look at the travel dataset in Slide 44. Given the categorical $D$ for a cultural trip, or an adventure trip, as age increases the categorical predictor gives opposite effects on the regression line. I.e. the slope and intercepts change dramatically; cultural trips become more popular with age, and opposite for adventure trips. Intepretation of $\beta_0$ is average amount spent on adventure when age is $0$, where adventure is set to $0$ in the categorical variable.
	\end{example}

	Depending on the travel group that these belong to, there is an different effect on the regression line. The indicator variable should be added to the model. We may then check whether the interaction term is significant with a t-test. 
	
	\subsection{Model Selection}
	
	If we have $n$ predictors for $n$ observations, then we get a perfect fit since our projection space can be the whole plane. We cannot just keep adding variables to our model, since we can overfit on the test set. We now move to prediction and predictive modeling; the first step to avoid overfitting is through model selection.\footnote{At the most basic level, linear models are a form of machine learning. Once we `learn' model parameters, we can predict $Y$ for a new dataset.}\\
	
	As we saw in multicolinearity, it is difficult to decide which predictors to include in a model. This general process is {\bf model selection}, also called {\bf variable selection}. What makes a model `best' depends on the purpose of the model; prediction, interpretation, etc. If interpretability is best, prediction accuracy is secondary, and fewer significant variables are best. For prediction, adding variables is important; more predictors lead to predictions with lower bias with larger variance. We consider some criteria for choosing possible subsets of $p$ predictors.
	
	\subsubsection{Adjusted $R^2$}
	
	Recall that as you increase the number of predictors, then the multiple coefficient of determination $R^2$ also increases. We therefore choose the smallest model that maximizes $R^2_{adj}$, but this may overfit and should be used with caution.
	
	\subsubsection{Akaike's Information Criterion}
	
	\begin{definition}
		{\bf Akaike's infromation criterion} is given by
		$$
			-2 \BB{\ell(\hat\beta, \hat \sigma^2) - (p+2)} 
		$$
		where $\ell$ is the log likelihood of the model.
	\end{definition}
	
	 A large $\ell$ will decrease the AIC, but too many parameters increase the AIC. We want to choose the model with the lowest AIC. Rewriting, we see the relationship
	 $$\ell = \f{n}2 \log(2\pi \sigma^2) - \f1{2\sigma^2}RSS \imp AIC \propto n \log\BB{\f{RSS}{n}} + 2p$$
	 
	 \subsubsection{Corrected Akaike's Information Criterion}
	 
	 AIC has the tendency of overfitting or some situations, particularly when the penalty $p+2$ or $2p$ is not strong enough. This happens with small samples or the number of parameters is a large fraction of the sample size. In this case, we use the following metric
	 \begin{definition}
	 	The {\bf corrected AIC} is written 
	 	$$
	 		AIC_C = AIC + \f{2(p+2)(p+3)}{n-p-1}
	 	$$
	 	and is preferred to the AIC when $\df{n}{p+2} \leq 40$.
	 \end{definition}
 	The `best' model is also the one with the lowest $AIC_C$.
	
	\subsubsection{Bayesian Information Criterion}
	\begin{definition}
		The {\bf Bayesian Information Criterion} is written 
		$$
		BIC = - 2 \ell + (p+2)\log(n)
		$$
	\end{definition}
	This penalizes parameters more than $AIC$, and therefore prefers simpler models than AIC. It can also be simplified as
	$$
		BIC \propto  n \log\BB{\f{RSS}{n}} + (p+2)\log(n)
	$$
	The model with lowest $BIC$ is preferred.
	
	\begin{example}
		From the lecture slides, we fit models with various predictors, and notice that a particular subset has lowest $AIC, AIC_c, BIC$ and high $R^2_{adj} = 93\%$, so we use this model.
	\end{example}

	\subsection{Stepwise Variable Selection}
	
	 For $n$ possible predictors, there are $2^n$ possible models, so we cannot practically try all possible combinations. We use {\bf forward stepwise selection}: we try the SLR $Y \sim X_i$, and choose the most significant variable. Then we add less and less significant variables $X_j$ in $Y \sim X_i, X_j$, until $BIC$ stops decreasing. Similarly, in {\bf backward stepwise selection} we can delete predictors one at a time from $Y \sim X_1, \ldots, X_p$ until $BIC$ is minimized. \\
	 
	 Both ways are equivalent to choosing the predictor with the lowest $p$-value. Adding variables with low $p$-value increases probability of type I error, but removing increases type II error. Type II error is `less controversial' than type I, so this method is preferred. Ideally, both forwards and backwards addition will give the same model, but in practice this often does not happen. To do the full form of {\bf stepwise variable selection}, we go both back and forth, adding and removing variables. Diagnostics after stepwise selection should also be done, and to note how much they change in comparison to before selection, but do not need to be published.\\
	 
	 While these are quite helpful, the estimated coefficients that we get from a post-selection model will actually be biased estimators. This can result in enlarged test statistics $t, F$ that are larger than they should be. We need to determine whether a model is reasonable for prediction purposes, that is validate it.
	 
	 \subsection{Bias-Variance Decomposition}
	 
	 So far we discussed inference: estimating true population relationships, and prediction: how well the fitted model predicts new data. Prediction is the basis of machine learning.\\
	 
	 In ML, the {\bf bias-variance tradeoff} is important. We first discuss the concept of learning and testing datasets. The {\bf training dataset} is used for model fitting, but the {\bf testing} dataset is used to check predictions. Training and and testing sets must be independent; samples must be partitioned between the two. {\bf Overfitting} to training data occurs when a model performs much worse on the test data. 
	 
	 \begin{definition}
	 	Suppose we want to predict an {\bf unobserved} $y_0$ at the test point $x_0$. Let $y_0 = f(x_0)$ be the true, possibly non-linear, relationship, and our linear prediction be denoted $\hat y_0$. Then the {\bf mean squared error} is given by
	 	$$
	 	MSE(x_0) = E_\tau [f(x_0) - \hat y_0]^2 = E_\tau [\hat y_0 - E_\tau(\hat y_0)]^2 +  [ E_\tau(\hat y_0) - f(x_0)]^2 = V(\hat y_0) + [ E_\tau(\hat y_0) - f(x_0)]^2
	 	$$
	 	where $\tau$ is the conditional training data, and the second term is the squared bias.
	 \end{definition}
 	In machine learning, the mean squared error is a commonly used loss function, measuring the deviation of prediction from training data. This decomposition becomes very useful in this context. Minimizing MSE minimizes bias or variance or both for $\hat y_0$ given training set $\tau$. Bias indicates how accurate predictions are, and variance gives how much predictions change from sample to sample. L.S. estimates are unbiased for the true model, but the variance can be very large when there are lots of predictors for limited observations.
	 
	 \subsection{Shrinkage Methods}
	 
	  Recall the purpose of model selection. When there are too many variables prediction variance increases, and interpretability suffers. We discussed stepwise variable selection, but this does not work when $n \leq p$. \\
	  
	  One idea is to apply some constraint that shrinks less important parameter estimates to $0$. {\bf Ridge regression} shrinks the coefficients by imposing a penalty on their size, but does {\bf not} make them $0$, and is {\bf not} used for variable selection. In ridge regression,
	  $$
	  	\hat \beta = \arg\min_\beta \left\{ \sum_i^n \BB{y_i - \beta_0 - \sum_{j=1}^p x_{i,j} \beta_j} + \lambda \sum_{j=1}^p \beta_j^2  \right\}
	  $$
	  which is equivalent to 
	  $$
	  	\arg\min_\beta \sum_i^n \BB{y_i - \beta_0 - \sum_{j=1}^p x_{i,j} \beta_j} \text{ and } \norm{\hat \beta}_2^2 \leq t,\, t\in \R
	  $$
	 	Ridge regression is used to estimate coefficients of models when the predictors are highly corelated. The additional penalty on the model adds a degree of bias, but reduces the high variance caused by multicolinearity: part of the bias-variance tradeoff. In MLR, we write
	 	$$RSS(\lam) = (Y - X\beta)'(Y - X\beta) + \lambda \beta'\beta$$
	 and minimizing the $RSS$ produces
	 $$\hat \beta = (X'X + \lambda I)^{-1}X'Y$$
	  
	  An important method for variable selection is a similar minimization subject to the {\bf LASSO: Least Absolute Shrinkage and Selection Operator}:
	  $$
	  \hat \beta = \arg\min_\beta \left\{ \sum_i^n \BB{y_i - \beta_0 - \sum_{j=1}^p x_{i,j} \beta_j} + \lambda \sum_{j=1}^p \abs{\beta_j}  \right\}
	  $$
	  This has no closed form solution, and must be found numerically. The minimal $\hat \beta$ gives values where many $\beta_j$ are $0$, and therefore less important, so we may remove the corresponding predictors. It is equivalent to
		$$
			\arg\min_\beta \sum_i^n \BB{y_i - \beta_0 - \sum_{j=1}^p x_{i,j} \beta_j} \text{ and } \norm{\hat \beta}_1 \leq t,\, t \in \R
		$$
	Lasso only selects $n$ variables, cannot select $p \geq n$ variables. Lasso can fail to do grouped selection of predictors with multicolinearity to reduce variance, and instead just selects one: it cannot select all of them, like ridge can.
	\begin{example}
		Choosing age categories, one particular category may be chosen, with predictor for other categories being $0$ due to LASSO shrinkage. However, we lose information, since other age ranges may be associated with other variables.
	\end{example}
	
	 We use a linear combination of shrinkage methods to fix this. A mixed regularization, the {\bf elastic net penalty}, was introduced using the strengths of both ridge and LASSO, with {\bf elastic net  mixing parameter $\alpha$}
	$$
		\lambda \BB{(1-\alpha) \sum_{i=1}^p \beta_i^2 + \alpha \sum_{i=1}^p \abs{\beta_i}} = \lambda \BB{ (1-\alpha) \norm{\beta}_2^2 + \alpha \norm{\beta}_1 } 
	$$
	In all cases, $\lam$ is chosen by cross validation, or can be chosen with the \verb*|glmnet| package in R.
	
	\newpage
	\section{June 13: Lecture 9}
	
	The final lecture for the assignment. We discuss assignment after model validation, and revisit multicolinearity. 
	
	\subsection{Model Validation}
	
	Model validation happens through analyzing how the model generalizes to test data. We discuss whether we overfit to training data, and how the model performs with new data. One independent test is often not enough to validate a model, since we may have leverage points or outliers. We need many test sets, but this is often not possible. We have two goals:
	\begin{enumerate}
		\item {\bf Model selection:} Estimating performances of different predictors to find the most predictive ones
		\item {\bf Model validation:} Estimating the prediction error on new data
	\end{enumerate}
	\begin{example}
		If our sample is $n=100$, we can partition the set into $k=10$ groups. The first 9 datasets may be used for model fitting, the 10th is used to check prediction error. Using every set of 10 datapoints as test set, but others as training set, we do \bf{cross-validation}. This is {\bf 10 fold cross validation}.
	\end{example}
	\begin{definition}
		Choosing different partitions of data set as training data, and the rest as training data, for different partitions, is {\bf cross validation}.
	\end{definition}
	Resampling methods allow us to classify or predict a response accurately, but we skip bootstrap for now. Cross validation is important for the assignment!\\
	
	{\bf Cross validation algorithm:}
	\begin{itemize}
		\item Randomly split the data into $k$ equal parts. 
		\item Fit the model with $k-1$ training parts, predict the outcomes for the last test part.
		\item use all $k$ parts as a test set.
		\item The prediction accuracy can be checked with mean absolute bias or {\bf mean squared error}. 
		\item The predictions an be plotted with observed values to check the accuracy of the estimates visually.
	\end{itemize}

	For cross validation, we will be using the \verb*|ols| code from \verb*|rms| package in \verb*|R|.
	
	\begin{definition}
		The estimator of the {\bf mean squared error} is given by
		$$
			MSE = \f{\sum (y_i - \hat y_i)^2}{n}
		$$
	\end{definition}

	\begin{example}
		In the slides, we construct model and do validation. We choose $\lam$ for regularization using cross-validation. See this week's R code.
	\end{example} 

	\subsubsection{Requirements for Assignment}
	Literature review/EDA, model fitting, diagnostics, variable selection. We only show the diagnostics for the very last model. Show only the steps that matter in this assignment. Improperly captioned and labelled tables lose marks: practice writing a real report. 
	
	\subsubsection{Clarifications}
	
	\begin{itemize}
		\item Issues may have started around variable transformations. Try showing $\sin^{-1}(\sqrt{y})$ is appropriate for the binomial case, when $\sigma^2 \propto E(Y)(1-E(Y))$.
		\item In the Box-Cox transformation we use the Newton-Raphson numerical method to obtain $\lambda$ for the most appropriate power transformation.
	\end{itemize}

	\section{June 15: Lecture 10}
	
	\subsection{Assignment Hints}
	
	When running \verb*|vif| in the project, we get a \verb*|GVIF| column. Recall when we calculate the variance inflation factor, we define vif for $X_i$ as $\df{1}{1-R_i^2}$ where $R_i^2$ is the coefficient of determination of $X_i \sim X_1, \ldots, X_n$. The GVIF has the exact same interpretation of VIF, but for the general case with categorical predictors. \\
	
	\Note LASSO in this case is bad, since it does not work well with many categorical variables.

	\subsection{Generalized and Weighted Least Squares}
	
	Inference for our model parameters require the Gauss-Markov assumption. The Gauss-Markov theorem assumes $V(\eps) = \sigma^2 I$. This is a strong assumption: instead of this we can assume $V(\eps) = \sigma^2 V$ where $V$ is a cov. matrix, and has non-zero off diagonal elements. Note there is covariance between $\eps_i, \eps_j$ and variance is non-constant.\\
	
	In this case we cannot minimize $(Y - X\beta)'(Y-X\beta)$. This will not provide the correct minimum since it assumes homoscedasticity. We minimize
	$$
		\arg \min_\beta (Y - X\beta)'V^{-1}(Y-X\beta)
	$$
	Since $\sigma^2V$ is a covariance matrix, it must be symmetric and positive definite. Then there exists an $n \times n$ symmetric matrix $K$ so that $K'K = KK = V$: it admits a square root. We define new variables 
	$$Z = K^{-1}Y, B=K^{-1}X, \gamma = K^{-1}\eps \text{ so that } Z = B\beta + \gamma$$
	First note $E(\gamma) = E(K^{-1} \eps ) = K^{-1}E(\eps) = 0$.
	Then 
	\begin{align*}
		V(\gamma) &= E\BB{(\gamma - E(\gamma))(\gamma - E(\gamma))'} = E\BB{\gamma\gamma'} \\
		&= K^{-1}E(\eps\eps')K^{-1} = K^{-1}\sigma^2V'K^{-1} = \sigma^2 I 
	\end{align*}
	So the transformed variables satisfy the Gauss-Markov assumptions. However, the matrix $V$ is very difficult to estimate. Minimizing the RSS for the transformed variables with GM assumptions,
	$$RSS(\beta) = \gamma'\gamma \eps'V^{-1}\eps = (Y-X\beta)'V^{-1}(Y-X\beta)$$
	and it can be shown that (fill in)\\

	Now assume in the case of generalized least squares, the covariance terms between elements are zero but variances are unequal. Then we may represent $\sigma^2V = \sigma^2W^{-1}$ where $W$ is the weight matrix. We therefore have {\bf weighted least squares}. Then $\hat \beta = (X'WX)^{-1}X'WY$ and $V(\hat \beta) = \sigma^2 (X'WX)^{-1}$. \\
	
	When homoescadicity is violated, the variance of $\eps$ also depends on $X$. (fill this in bro)
	
	\subsection{Polynomial Regression}
	
	In linear regression, we mean linear with respect to $\beta$: linear in parameters. This makes sense from a statistical point of view, since we minimize $RSS$ which is wrt $\beta$. The relationship can be non-linear, as long as
	$$
		Y = \beta_0 + \beta_1 \phi_1(X_1) + \beta_2 \phi_2(X_2) + \ldots + \beta_p\phi_p(X_p)
	$$
	Based on this equation, we can fit $y = \beta_0 + \beta_1 x + \beta_2 x^2 + \eps$. Or $y = \beta_0 + \beta_1 x + \beta_2 x^2 + \beta_{1,1} x_1^2 + \beta_{2_2} x_2^2 \eps + \beta_{1,2} x_1x_2$.
	
	\subsection{Generalized Linear Model}
	
	We go deep into GLM and GAM in STA303. Up to this point, we have considered $Y$ to be a continuous random variable.	Assume $Y$ is a binary variable. If $Y = 0,1$, then $0<E(Y \mid X) < 1 \imp 0 < \beta_0 + \beta_1 X < 1$. However if we take $\log(E(Y \mid X))$ as our response, then $E(Y\mid X) \in (0,1)$ give $-\infty < \beta_0 + \beta_1 X < 0$. Then the ratio $\log \BB{\f{E(Y\mid X)}{1 - E(Y \mid X)}} = \beta_0 + \beta_1 X$ has a linear relationship with respect to the parameters. Calculating, we can find
	$$
		E(Y \mid X) = \f{\exp(\beta_0 + \beta_1 X)}{1 + \exp(\beta_0 + \beta_1X)}
	$$
	where this is the {\bf cdf of the logistic distribution}! This is {\bf logistic regression}.
	We call the odds $\Omega = \f{E(Y\mid X)}{1 - E(Y \mid X)}$
	\begin{definition}
		A {\bf generalized linear model} is a model so that there exists a {\bf link} function $\phi$ where $$\phi(E(Y \mid X))$$ is a linear function of parameters $(\beta_i)_{0 \leq i \leq p}$. In a {\bf linear model}, $\phi = I$. 
	\end{definition}
		
	\subsection{Generalized Additive Model}
	
	An extension of GLM. 
	\begin{definition}
		An {\bf additive model} is a model so that 
		$$
			Y = \beta_0 + \sum_{j=1}^p f_j(X_j) + \eps.
		$$
		we choose {\bf any} function $f_j$. In its basic form, the additive model will do poorly
	\end{definition}

	\begin{definition}
		A {\bf generalized additive model} is a model is a a model so that there exists a {\bf link} function $\phi$ where $$\phi(E(Y \mid X))$$ is an additive model of $f_j$.
	\end{definition}

	We move to more and more flexible models.

	
		
		
		
			
	
		
		

\end{document}